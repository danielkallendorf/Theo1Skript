% !TeX spellcheck = de_frami_dfew
\documentclass[10pt,a4paper]{scrartcl}
\usepackage[utf8]{inputenc}
\usepackage[german]{babel}
\usepackage{amsmath}
\usepackage{amsfonts}
\usepackage{amssymb}
\usepackage{amsthm}
\usepackage{graphicx}
\usepackage{tikz,pgf}
\usepackage{pgfplots}
\usetikzlibrary{calc}
\usetikzlibrary{decorations}
\usepackage{yfonts}
\usepackage[arrow, matrix, curve]{xy}
\usepackage{wrapfig}

\newcommand{\gdw}{\Leftrightarrow}
\newcommand{\N}{\mathbb{N}}
\newcommand{\Z}{\mathbb{Z}}
\newcommand{\Q}{\mathbb{Q}}
\newcommand{\R}{\mathbb{R}}
\newcommand{\C}{\mathbb{C}}
\newcommand{\F}{\mathbb{F}}
\newcommand{\impl}{\Rightarrow}
\newcommand{\la}{\lambda}
\newcommand{\al}{\alpha}
\newcommand{\SeiV}{Sei $V$ ein $K$-Vektorraum }
\newcommand{\summeA}{\sum_{i\in I}}
\newcommand{\summeB}{\sum_{i=1}^n}
\newcommand{\ol}[1]{\overline{#1}}
\newcommand{\norm}[1]{\|#1\|}
\newcommand{\dif}[2]{\frac{\partial #1}{\partial #2}}
\newcommand{\mdif}[3]{\frac{\partial^#1 #2}{\partial #2^#3}}


\theoremstyle{remark}
\newtheorem*{bem}{Bemerkung}

\theoremstyle{definition}
\newtheorem*{exm}{Beispiel}
\newtheorem*{defn}{Definition}

\theoremstyle{plain}
\newtheorem*{theorem}{Satz}
\newtheorem*{mth}{Mehtode}


\catcode`\"=\active
\AtBeginDocument{
	\newcommand{"}[1]{^{(#1)}}
}

\author{Daniel Kallendorf}
\title{Theoretische Physik 1}
\date{}

\begin{document}
\maketitle
\begin{center}
	Mitschrift der Vorlesung Theoretische Physik I \\
	WS 2015/16 bei Prof. Alber
\end{center}
\tableofcontents
\newpage




\section{Das Zweikörperproblem im Rahmen der Newtonschen Mechanik}

\subsubsection*{Problemstellung}
\begin{center}
\begin{tikzpicture} %[scale=0.5]
	\coordinate  (m1) at (-3,1);
	\coordinate  (m2) at (0,2);
	\coordinate  (0) at (0,0);
	\draw[fill=black] (0) circle(0.05) node[below]{$0$};
	\draw[fill=black] (m1) circle(0.05) node[below]{$m_1$};
	\draw[fill=black] (m2)circle(0.05) node[above]{$m_2$};
	\draw[->] (0)--(m1);
	\draw[->] (0)--(m2);
	\node[left]  at (0,1) {$\vec x_2(t)$};
	\node[below]  at (-1.5,0.5) {$\vec x_1(t)$};
	\draw[->](m1)--(m2);
	\node[above]  at (-2,2){$\vec r(t)=\vec x_1(t)-\vec x_2(t)$};
\end{tikzpicture}
\end{center}
\subsubsection*{Newtonsche Axiome}
\begin{enumerate}
	\item kräftefreie Massenpunkte bewegen sich geradlinig gleichförmig:\\
	Inertialsystem
	\item $m\ddot{\vec x}(t)=\vec F(t)$
	\item Wechselwirkung: actio=reactio
\end{enumerate}
\subsubsection*{Bemerkungen}
\begin{itemize}
	\item $\vec F(t)$ beschriebt Kraftkurve
	\item $\vec x_1(t)$ beschreibt Bahnkurve (ohne Zeitinformation: Bahnkurve)
	\item Spezielle Kraftfelder:\\
		$\vec K(\vec x,\dot{\vec x},t)\rightarrow \vec F(t)=\vec K(\vec x(t),\dot{\vec x}(t),t)$\\
		\underline{Newton}: $\vec K(\vec x(t),\dot{\vec x}(t),t)=-\gamma \dfrac{\vec x_1-\vec x_2}{\norm{\vec x_1-x_2}^3}m_1^{(T)}m_2^{(s)}$\\
		$m^{(s)}$: schwere Masse\\
		$m^{(t)}$: träge Masse\\
		\fbox{$m^{(T)}\equiv m^{(s)}$}
\end{itemize}
\begin{align*}
	\vec K(\vec x_1,\vec x_2)=\dfrac{\vec r(t)}{\norm{\vec r(t)}}f(\norm{\vec r(t)})&=-\vec{\nabla}_{\vec x_1}U(\norm{\vec x_1-\vec x_2})\\
	\Rightarrow \int_{\xi_1|\vec x_1^A}^{\vec x_1^B}\vec K(\vec x_1,\vec x_2)\left(\dfrac{d\vec x_1}{d\lambda}\right)d\lambda&=-U(\norm{\vec x_1^B-\vec x_2})-U(\norm{\vec x_1^A-\vec x_2})
\end{align*}
mit  Weg $\xi:\la\rightarrow \vec x_1(\la)$ und wegunabhängiger konservativer Kraft
\subsection{Erhaltungssätze}
\begin{itemize}
	\item abgeschlossenes System: (action=reactio)\\
		\begin{align*}
			\left. {m_1\ddot{\vec x}_1(t)=\vec F_{12}(t)}\atop{m_1\ddot{\vec x}_1(t)=\vec F_{12}(t)}\right\}&\dfrac{d}{dt}\underbrace{(m_1\dot{\vec x}_1+m_2\dot{\vec x}_2)(t)}_{:=\vec P(t)\text{ Impuls }=M\dot{\vec{R}}(t)}=0\\
			\underbrace{(m_1+m_2)}_{:=M}\vec R(t)&:=m_1\vec x_1
			(t)+m_2\vec x_2(t)
		\end{align*}
		\begin{align*}
			\begin{array}{cc}
			\vec x_1(t)&=\vec R(t)+\frac{m_2}{M}\vec r\\
			\vec x_2(t)&=\vec R(t)+\frac{m_1}{M}\vec r\\
			\end{array}\Leftrightarrow\begin{array}{cc}
			\vec R(t)&=\dfrac{m_1\vec x_1(t+m_1\vec x_2(t))}{M}\\
			\vec r(t)&=x_1(t)-x_2(t)
			\end{array}
			\end{align*}
			\item Zentralkraftfeld\\
			$\mu\ddot{\vec r}(t)=-\dfrac{\vec r(t)}{\vec r(t)}f(r(t)):$ allgemeine Zentralkraft (z.B. Gravitation)\\
			$\rightarrow $ Drehimpulserhaltung\\
		\begin{align*}
			\vec L&:=m_1\vec x_1(t)\times \dot{\vec x}_1(t)+m_1\vec x_2(t)\times\dot{\vec x}_2(t)&\text{ zu Zeigen: }\dot{\vec L}(t)=0\\
			&=\vec R\times \vec P+\mu\vec r\times\dot\vec r\\
			\dot\vec L(t)&=\underbrace{\dot\vec R\times\vec P}_{=0}+\underbrace{\vec R\times\dot\vec P}_{=0}\mu\vec r\times\ddot\vec r+\underbrace{\mu\dot\vec r\times\dot\vec r}_{=0}=0
		\end{align*}
		Relativbahndrehimpuls $L_{rel}$:
		\begin{itemize}
			\item $\vec L_{rel}\cdot \vec r(t)=0\rightarrow $ ebene Dynamik
			\item $\dfrac{d}{dt}\mu\vec r\times\dot\vec r=0$ Flächensatz
		\end{itemize}
	\item konservative Kraftfelder
		\begin{align*}
			\left.\begin{array}{ccccc}
			\dot{\vec x}_1m_1\ddot{\vec x}_1&=&\vec F_{12}(t)&=&-\vec{\nabla}_{\vec x_1}U(\norm{\vec x_1-\vec x_2})\\
			\dot{\vec x}_2m_2\ddot{\vec x}_2&=&-\vec F_{12}(t)&=&-\vec{\nabla}_{\vec x_2}U(\norm{\vec x_1-\vec x_2})
			\end{array}\right\}t\\
		\end{align*}
		Die Funktion U darf nur durch $x_1$ und $x_2$ von der Zeit abhängen
		\begin{align*}
			m_1\dot{\vec x}_1\ddot{\vec x}_2+m_2\dot{\vec x}_2\ddot{\vec x}_2&=-\dot{\vec x}_1\vec{\nabla}_{\vec x_1}U-\dot{\vec x}_2\vec{\nabla}_{\vec x_2}U\\
			\dfrac{d}{dt}\left(\dfrac{m_1}{2}\dot{\vec x}_1^2+\frac{m_2}{2}\dot{\vec x}_2^2\right)&=-\dfrac{d}{dt}U(\vec x_1(t)-\vec x_2(t))
		\end{align*}
		\begin{equation*}
			\Leftrightarrow \frac{d}{dt}\underbrace{\left(\dfrac{m_1}{2}\dot{\vec x}_1^2+\dfrac{m_2}{2}\dot{\vec x}_2^2+U(\vec x_1(t)-\vec x_2(t))\right)}_{:=E}=0
		\end{equation*}
\end{itemize}


\subsection{Bestimmung der Bahnkurven}

\subsubsection*{Erhaltungsgrößen}

\begin{itemize}
	\item abgeschlossenes System: Impulserhaltung \\
		$\dot{\vec P}=0, \vec R(t)=\vec R(t_0)+\dfrac{\vec P}{M}(t-t_0)$
	\item Zentralkraftfeld: Drehimpulserhaltung\\
		$\vec L=m_1\vec x_1\times\dot{\vec x}_1+m_2\vec x_2\times\dot{\vec x}_2=\vec L_{cm}+\vec L_{rel}$\\
		$\vec L_{rel}=\mu\vec r\times\dot{\vec r}$
	\item konservative Kraft: Energieerhaltung
\end{itemize}


\subsubsection*{Ansatz}
Aus $\vec L_{rel}\cdot \vec r=0$ folgt, dass die Bewegung in einer Ebene stattfindet.
\begin{itemize}
	\item Wir definieren $\vec L_{rel}\vec e_z$.
	\begin{align*}
		x&=r\cos\varphi&\dot x&=\dot r\cos\varphi-r\sin\varphi\dot{\varphi}\\
		y&=r\sin\varphi&\dot y&=\dot r\sin\varphi+r\cos\varphi\dot{\varphi}
	\end{align*}
	\begin{equation}
		L_{rel}=\mu(x\dot y-\dot xy)=\mu\left[r\cos\varphi\left(\dot r\sin\varphi+r\cos\varphi\dot{\varphi}\right)-\left(\dot r\cos\varphi-r\sin\varphi\dot{\varphi}\right)r\sin\varphi\right]=\mu r^2\dot{\varphi}
	\end{equation}
	Somit erhalten wir
	\begin{align*}
		\dot{\varphi}&=\frac{L_{rel}}{\mu r^2}\\
		\varphi(t)-\varphi_0&=\int_{t_0}^{t}dt'\frac{L_{rel}}{\mu r^2(t')}
	\end{align*}
		\item $E_{rel}=\frac{\mu}{2}\dot{\vec r}^2+U(r)$
	\begin{align*}
		\dot{\vec r}^2&=\dot x^2+\dot y^2=\dot r^2+r^2\dot{\varphi}^2\\
		E_{rel}&=\frac{\mu}{2}\dot r^2\frac{L_{rel}}{\mu^2r^4}+U(r)\\
		&=\frac{\mu}{2}\dot r^2+\underbrace{\frac{L_{rel}^2}{2\mu r^2}+U(r)}_{=U_{eff}(r)}\\
		\dot r&=\sqrt{\frac{2}{\mu}(\underbrace{E_{rel}-U_{eff}(r)}_{\ge 0})}=\frac{dr}{dt}
	\end{align*}
	\begin{equation*}
		t-t_0=\pm\int_{r(t_0)}^{r(t)}dr\sqrt{\frac{\mu}{2}(E_{rel}-U(r))}
	\end{equation*}
		mit $+$ für $r(t)>r(t_0)$, $-$ für $r(t)<r(t_0)$, sonst $0$.
\end{itemize}


\subsubsection*{Bestimmung des Orbits}
\[\dfrac{\dot{\varphi}}{\dot r}=\frac{d\varphi}{dr}=\pm\frac{L_{rel}}{\mu r^2\sqrt{\dfrac{2}{\mu}(E_{rel}-U(r))}}\]
\[\varphi(r)-\varphi(r_0)=\pm\int_{r_0}^{r}dr'\frac{L_{rel}}{r'^2\sqrt{2\mu(E_{rel}-U(r'))}}\]
\begin{figure}[bh]
	\begin{tabular}{|ccc|}
		\hline&
		\begin{tikzpicture}[scale=0.5]
			\coordinate (Z) at (0,0);
			\draw[fill] (Z) circle(0.1) node[above]{$m_1$};
			\draw [dashed](Z) circle(3);
			\draw(Z) ellipse(3 and 1);
			\draw[rotate around={60:(Z)}] (Z) ellipse (3 and 1);
			\draw[rotate around={120:(Z)}] (Z) ellipse (3 and 1);
		\end{tikzpicture}&
		\begin{tikzpicture}[scale=0.5];
			\coordinate (Z) at (0,0);
			\draw[fill] (Z) circle(0.1) node[above]{$m_1$};
			\draw[dashed] (Z) circle (1);
			\draw plot[domain=-1.7:1.4] ({-cosh(\x)+2},{sinh(\x)});
			\draw[dashed](2,0)--(-1,-3) node[right]{Asymptote};
		\end{tikzpicture}\\
		$E_0$ nicht möglich&
		$E_0'$ gebundene Bewegung&$E_0''$ ungebundene Bewegung\\
		\hline
	\end{tabular}
	\caption{Bahnkurven}
	\vspace{-1cm}
\end{figure}

%VL 20.10.2015
\begin{figure}[b]
	\begin{tikzpicture}[scale=0.8]
		\coordinate (0) at (0,0) node[left] at (0) {$0$};
		\coordinate (E0) at (-0.15,-2) node[left] at (E0) {$E_0$};
		\coordinate (E1) at (-0.15,-1) node[left] at (E1) {$E_0'$};
		\coordinate (E2) at (-0.15,1) node[left] at (E2) {$E_0''$};
		\draw[thick,->] (-0.15,0)--(4,0) node[right]{$r$};
		\draw[thick,->] (0)--(0,2) node[left] {$U_{eff}$};
		\draw[thick] (0)--(0,-2.3);
		\draw[dashed] (E0)--(4,-2);
		\draw[dashed] (E1)--(4,-1);
		\draw[dashed] (E2)--(4,);
		\draw plot[domain=0:6.5,smooth] ({0.6*\x},{1.5*(3-(\x+1.3)^2))/(e^(1.1*\x)});
	\end{tikzpicture}
	\caption{Energieschema}
\end{figure}


\begin{defn}
	Ein \emph{Funktional} ordnet jeder möglichen Bahnkurve eine Zahl zu.\\
\end{defn}
Man kann also Funktionale finden, sodass die Extrema auf die Newtonschen Bewegungsgleichungen führen.\\
Es ergibt sich ein Variationsproblem im $\infty$-dimensionalen Raum der Bahnkurve unendlich vieler Punkte.
\\

\newpage




\section{Das Hamiltonsche Prinzip der klassischen Mechanik}
Aus dem Variationsprinzip lassen sich die Newtonschen Bewegungsgleichungen herleiten. Dies ermöglicht
\begin{itemize}
	\item die Vereinheitlichung mechanischer Prozesse in ihrer Beschreibung
	\item die Beschreibung von Zwangskräften (Wände, Schiene, etc)
\end{itemize}
Dabei ``bewerten'' die Funktionale die möglichen Bahnkurven. Ein Extremum des Bewegungsfunktionals liefert also eine Bewegungsgleichung(= Lösung der Newton Gleichungen)
\subsection{Funktionale und Funktionalableitungen}
Es gilt $x_1,x_2\in\R^{3N}$, da jeder der N Massenpunkte je 3 Koordinaten hat.\\
Wir betrachten nun $B_{x_1,x_2}$, die Menge aller stückweise stetigen und differenzierbaren Bahnkurven von $x_1$ (zum Zeitpunkt $t_1$) nach $x_2$ (zu $t_2$).\\
Und das Funktional $F:B_{x_1,x_2}\rightarrow \R,\gamma\mapsto F[\gamma]$ (Punktweise Auswertung).

\paragraph*{Stetigkeit}F ist stetig, an $\gamma_0$, wenn das $\delta-\epsilon$-Kriterium gilt:\\
Für alle $\epsilon>0$ und für alle $h$ mit $\norm{h}<\delta$, existiert ein $\delta>0$, sodass
\[|F[\gamma_0+h]-F[\gamma_0]|<\epsilon\]
Wobei man $h$ auch schreiben kann, als
\[\norm{h}=\max\limits_{t_1\leq t\le t_2}\left(\sum_{i=1}^{N}\sqrt{\vec{h}_i(t)\cdot\vec{h}_i(t)}\right)+\max\limits_{t_1\le t\le t_2}\left(\sum_{i=1}^{N}\sqrt{\dot{\vec{h}}_i(t)\cdot \dot{\vec{h}}_i(t)}\right)\]


\subsubsection*{Funktionalableitungen}
Ein Funktional heißt \emph{ableitbar}, wenn eine Beste Lineare Approximation existiert:
\[\exists F_{\gamma_0}':F[\gamma_0+h]-F[\gamma_0]=F'_{\gamma_0}[h]+O(\norm{h}^2)\]

\subparagraph*{lokales Funktional}
\[F_1[\gamma]\sum_{i=1}^{N}\sqrt{\vec x_i(t_0)\cdot\vec c_i(t_0)}\]
mit festem $t_0$ und $t_1\le t_0\le t_2$ gilt für
\[S[\gamma]=\int_{t_1}^{t_2}dt\quad L(\{\vec x _i\}(t),\{\dot{\vec x} _i \}(t),t)\]
dass
\[S[\gamma+h]-S[\gamma]=S'_\gamma[h]+O(\norm{h^2}^2)\]
\[\text{mit}\quad S_\gamma'[h]=\sum_{i=1}^{3N}h_i(t)\left.\dif{L}{\dot x_i}\right|_{t_1}^{t_2}+\sum_{i=1}^{3N}\int_{t_1}^{t_2}dt\quad h_i(t)\left\{\dif{L}{x_i}-\frac{d}{dt}\dif{L}{\dot x_i}\right\}_{\gamma(t)}\]

\begin{proof}[Beweisidee]
	\begin{align*}
		S[\gamma+h]&=\int_{t_1}^{t_2}dt L\big(x_i(t)+h_i(t),\dot x_i(t)+\dot h_i(t),t\big)\\
		&=S[\gamma]+\int_{t_1}^{t_2}dt\left\{\right.\dif{L}{x_i}\left|_\gamma h_i(t)+\left.\dif{L}{\dot x_i}\right|_\gamma\dot h_i(t)\right\}+O(\norm{h}^2)
		\intertext{Wir verwenden die Produktregel: $\left.\dif{L}{\dot x_i}\right|_\gamma\dot h_i(t)=\left.\frac{d}{dt}\dif{L}{\dot x_i}\right|_\gamma h_i(t)-h_i(t)\left.\frac{d}{dt}\left(\dif{L}{\dot x_i}\right)\right|_\gamma$}
		&=S[\gamma]+\left.\sum_{i=1}^{3N}h_i(t)\left.\left(\dif{L}{\dot x_i}\right)\right|_\gamma\right|_{t_1}^{t_2}+\sum_{i=1}^{3N}\int_{t_1}^{t_2}dth_i(t)\left\{\left.\dif{L}{x_i}\right|_\gamma-\frac{d}{dt}\left.\left(\dif{L}{\dot x_i}\right)\right|_\gamma\right\}
	\end{align*}
\end{proof}
Da das bei $S'_\gamma=0$ Extrema liege, kann man Bewegungsgleichungen formulieren: Die \emph{Euler-Lagrange-Gleichungen}
\[\dfrac{d}{dt}\left.\left(\dif{L}{\dot x_i}\right)\right|_\gamma-\left.\dif{L}{x_i}\right|_\gamma=0\]



%VL 22.10.2015
\subsection{Das Hamiltonsche Prinzip}
Für
\[L\big(\{x_i\},\{\dot x _i\},t\big)=\sum_{i=1}^{N}\left\{\frac{m_i}{2}\dot{\vec x}_i\cdot\dot{\vec x}+\vec F_i(t)\cdot \vec x_i\right\}\]
wobei $\vec F_i(t)$ vorgegebene Kraftkurven sind, gilt, dass die EL-Gl äquivalent zu den Newtonschen Bewegungsgleichungen ($m_i\ddot{\vec x_i}(t)=\vec F_i(t)$) sind.
\begin{proof}
	Es gilt:
	\begin{align*}
	\vec{\nabla}_{x_i}L&=\vec F_i(t)&\vec{\nabla}_{\dot x_i}L&m_i\dot{\vec{ x}}_i
	\end{align*}
	Somit also
	\[\underbrace{\left.\frac{d}{dt}\nabla_{\dot{\vec{x}}_i}L\right|_\gamma}_{\text{(Newton Gleichungen)}}=\underbrace{\left.\nabla_{\vec{x}_i}L\right|_\gamma}_{\text{(Euler-Langrange-Gleichungen)}}\]
\end{proof}
\begin{bem}
	\begin{itemize}
		\item Kraftkurven sind vorgegeben.
		\item $S[\gamma]=\int_{t_1}^{t_2}dt L\big(x(t),\dot x(t),t\big)$
		\item $h^2(t)\equiv \delta x_i(t)$, eine Virtuelle Verschiebung.\\
		Es handelt sich nicht um eine physikalische Bewegung sondern um einen rein mathematische infinitesimale Verschiebung eines Teilchens, die ohne verzögerung (instantan) stattfindet.\\
		(Unter beachtung der Zwangsbedingungen)
		\item Die äquivalenz stimmt nicht vollständig, da die Newton-Gleichungen eine Anfangsrandwertproblem darstellen, das Variationsprinzip/ Hamiltonsches Prinzip ein Randwertproblem darstellt:
	\begin{itemize}
		\item Newton: $\vec x_i(t,x_1,\dot x_1)$
		\item Hamilton: $\vec x_i(t,x_1,x_2)$
	\end{itemize}
	Die Äquivalenz gilt also, wenn dem Ziel bijektiv eine Anfangswert zugeordnet werden kann.
	\end{itemize}
\end{bem}

Sei nun $S[\gamma]=\int_{t_1}^{t_2}dt L\big(x(t),\dot x(t),t\big)$ mit
\[\underbrace{\delta\vec x(t_1)}_{\vec h(t_1)}=\delta\vec x(t_2)\]
Da Hamilton Prinzip selektiert nun $\gamma_0$
\[\delta S[\gamma_0]=S'_{\gamma_0}[h]=0\]
ist äquivalent zu den E-L-Gl:
\[\dfrac{d}{dt}\left.\left(\dif{L}{\dot x_k}\right)\right|_{\gamma_0}-\left.\dif{L}{x_k}\right|_{\gamma_0}=0\]
und den Newton-Gleichungen:
\[L=\sum_{i=1}^{N}\frac{m_i}{2}\dot{\vec{x}}^2+\sum_{i=1}^{N}\vec F(t)\vec x\]

\begin{bem}
	Geschwindigkeitsabhängige Potenziale\\
	Es gilt für:
	\[\vec F_i(t)=-\left(\vec{\nabla}_{\vec x_i}V-\frac{d}{dt}\left(\vec{\nabla}_{\dot{\vec{x}} _i}V\right) \right)(\{x(t)\},\{\dot x(t)\},t)\]
	und es ergibt sich
	\[L=\sum_{i=1}^{N}\frac{m_i}{2}\dot{\vec{x}}_i^2-V(\{x(t)\},\{\dot x(t)\},t)\]
\end{bem}
\begin{proof}
	z.Z.: $\delta\int_{t_1}^{t_2}dt\sum_{i=1}^{N}\vec F_i(t)\vec x_i(t)\overset{!}{=}-\delta\int_{t_1}^{t_2}dtV(\{x(t)\},\{\dot x(t)\},t)$\\
	Wir beginnen mit der rechten Seite:
	\begin{align*}
		\delta\int_{t_1}^{t_2}dtV(\{x(t)\},\{\dot x(t)\},t)&=\int_{t_1}^{t_2}dt\left\{V(\delta\int_{t_1}^{t_2}dtV(\{x(t)\},\{\dot x(t)\},t))\right\}\\
		&\quad+\sum_{i=1}^{N}\delta\vec x_i(t)\vec{\nabla}_{x_i}V+\sum_{i=1}^{N}\delta\dot{\vec{x}}_i(t)\vec{\nabla}_{\vec x_i}V+O(\norm{h}^2)\\
		\Rightarrow \delta\vec{x}_i(t)\vec{\nabla}_{\vec x_i}V_i&=\frac{d}{dt}\left(\Delta x_i\vec{\nabla}_{\dot{\vec{x}}_i}V\right)-\delta x_i(t)\frac{d}{dt}\left(\vec{\nabla}_{x_i}V\right)\\
		\Rightarrow \delta\int_{t_1}^{t_2}dtV(\{x(t)\},\{\dot x(t)\},t)&=\int_{t_1}^{t_2}dt\sum_{i=1}^{N}\left(\delta\vec x_i()\cdot\left[\vec{\nabla}_{\vec x_i}V-\frac{d}{dt}\vec{\nabla}_{\dot{\vec{x}}_i}V\right]+\underbrace{\frac{d}{dt}\left[\delta\vec x_i(t)\cdot\nabla_{\vec x_i}V\right]}_{\delta\vec x_i(t_1)=\delta\vec x_i(t_2)=0}\right)\\
		&=-\int_{t_1}^{t_2}dt\quad\vec F_i(t)\cdot\Delta x_i(t)\\
		&\equiv -\delta\int_{t_1}^{t2}dt\quad\vec F_i(t)\cdot \vec x_i(t)\qedhere
	\end{align*}
\end{proof}


\subsubsection*{Mechanische Eichfreiheit}
Aus einer Lagrange-Funktion lassen dich die Newtonschen Bewegungsgleichungen ableiten (nicht bijektiv).

\begin{proof}
	Sei $L_1(\{x(t)\},\{\dot x(t)\},t)$. Und sei
	\[L_2(\{x(t)\},\{\dot x(t)\},t)=L(\{x(t)\},\{\dot x(t)\},t)+\dif{f}{t}(\{x\},t)+\sum_{i=1}^{N}\dot{\vec{x}}_i\cdot\vec{\nabla}_{\vec x_i}f(\{x\},t)\]
	mit einer beliebigen (differenzierbaren) Funktion f.\\
	Dann führe beide Lagrange-Funktionen auf die selben E-L-Gln, denn
	\begin{align*}
		S[\gamma]&=\int_{t_1}^{t_2}dtL_2(\{x(t)\},\{\dot x(t)\},t)=\int_{t_1}^{t_2}L_1(\{x(t)\},\{\dot x(t)\},t)+\int_{t_1}^{t_2}dt\frac{d}{dt}f(\{x\},t)\\
		&=\dif{f}{t}(\{x(t)\},t)+\sum_{i=1}^{N}\dot{\vec x}_i(t)\cdot\left(\vec{\nabla}_{\vec x_i}f\right)(\{x(t)\},t)\\
		\partial S[\gamma]&\equiv S'_\gamma[h]=\delta\int_{t_1}^{t_2}dt\quad L(\{x(t)\},\{\dot x(t)\},t)\\
		\frac{d}{dt}f(\{x(t)\},\{\dot x(t)\},t)&=\dif{f}{t}(\{x(t)\},t)+\sum_{i=1}^{N}\dot{\vec{x}}_i(t)\cdot(\vec{\nabla_{\vec{x}_i}}f)(\{x(t)\},t)
	\end{align*}
\end{proof}
Somit hat also $L$ keine ``direkte'' physikalische Relevanz.\\
Daraus folgt jedoch: die Äquivalenzklasse der Lagrange-Funktionen
\[[L]=\{L_2|\exists f:L_2=L+f\}\]
Welche Äquivalent zu den E-L-Gln ist.
\subsubsection*{Analyse der Extrema von $[\gamma]$}
Die Frage, ob ein Extremum von $S[\gamma]$ ein Minimum, Maximum oder Sattelpunkt ist gibt uns $\delta^2S[\gamma]$
\[S[\gamma+\delta x]=S[\gamma]+S'_\gamma[\delta x]+O(\norm{\delta x}^2)\]
\begin{exm}
	Eine eindimensionale Dynamik mit geschwindigkeitsunabhängiges Potenzial und $\gamma:t\mapsto \vec x(t)$
	\[L(x,\dot x,t)=\frac{m}{2}\dot{x}^2-V(x,t)\]
	\begin{align*}
		S[\gamma+\delta x]&=\int_{t_1}^{t_2}dt\left(\frac{m}{2}[\vec x(t)+\delta \dot x(t)]^2-V(x(t)+\delta\dot x(t),t)\right)\\
		&=\underbrace{\int_{t_1}^{t_2}dt\left(\frac{m}{2}\dot{x}(t)^2-V\big(x(t),t\big)\right)}_{S[\gamma]}
		+\underbrace{\int_{t_1}^{t_2}dt\left(\underbrace{\frac{m}{2}\cdot 2\dot x(t)\delta\dot x(t)}_{=\frac{d}{dt}(\delta x\dot x)-\Delta x\frac{d}{dt}\dot x}-\delta x(t)\dif{V}{x}(x8(t),t)\right)}_{:=S'_\gamma[\delta x]\equiv\delta S[\gamma]}\\
		&\quad+\underbrace{\int_{t_1}^{t_2}dt\left(\frac{m}{2}\delta\dot x(t)^2-\frac{1}{2}\dif{^2V}{x^2}(x(t),t)\right)}_{:=\delta^2S[\delta x]}+O(\norm{\delta x}^3)
	\end{align*}
	wobei $\delta S[\gamma]=0$ da wir ein Extremum betrachten.\\
	\\
	Sei $|t_2-t_1|$ klein, dann ist für alle $\delta x$ $\delta^2 S[\delta x]\ge 0$ weil $???>0$.\\
	Für $|t_2-t_1|$ wachsen, existiert ein anderes Extremum mit $\delta^2 S[\delta x]<0$.
\end{exm}

\emph{Hier fehlt etwas}
%VL 27.10.2015
\begin{itemize} 
	\item $x^i(q,t)\leftrightarrow q^i(x,t)$ sei angenommen
	\[S\left[\gamma\right]=\int_{t_1}^{t_2}dt L(x(t),\dot c(t),t)=\int_{t_1}^{t_2}dt \underbrace{L\left(x(q,t),\left\{\frac{\partial x^i}{\partial t}+\sum_{j=1}^{N}\dot{\vec q}_j\cdot\vec{\nabla}_{\vec{q}_j}x^i\right\},t\right)}_{:=\tilde{L}(q,\dot q,t)}\]
	$\rightarrow $ Euler-Lagrange Gleichung für $i=1,...,3N$
	\[\dfrac{d}{dt}\frac{\partial\tilde L}{\partial \dot q_i}=\frac{\partial \tilde{L}}{\partial q_i}\]
\end{itemize}


\subsection{Symmetrie und Erhaltungsgrößen} %2.3)
Ausgangspunkt: $\dfrac{d}{dt}\frac{\partial L}{\partial \dot q_i}=\frac{\partial L}{\partial q_i}$ Euler-Lagrange für $L(q,\dot q,t)$.\\
Sei $\frac{\partial L}{\partial q_1}=0$, d.h.:
\begin{itemize}
	\item Transformation $\tau\epsilon$
	\item $L$ unabhängig von $q_1$; $\tau_{\epsilon^i} Q^i=q^i+\epsilon\delta_{i1}$
	\item $L$ invariant unter $\tau_\epsilon$
	\item $\tau\epsilon$ transformiert Lösungen der $ELG$ wieder in Lösungen (kontinuierliche Symmetrie)
\end{itemize}
Es folgt daraus, dass auch $\dfrac{d}{dt}\dfrac{\partial L}{\partial \dot q_1}=0\Leftrightarrow p_1=\dfrac{\partial L}{\partial \dot q_1}(q(t),\dot q(t),t)$ berechent entlang einer Lösung der E-L-Gleichung ist Zeitlich konstant\\
$\rightarrow $Erhaltungsgröße\\
$p_1:=$kanonischer Impuls von $q_i$\\
Also ``kontinuierliche Symmetrie impliziert Erhaltungsgrößen''


\subsubsection{Nöther Theorem}
Betrachte folgende kontinuierliche Transformation:
\begin{align*}
	Q^m&=q^m+\epsilon\psi^m(q,\dot q,t)+O(\epsilon^2)\\
	T&=t+\epsilon\phi(q\dot q,t)+O(\epsilon^2)
\end{align*}

\begin{center}
mit $\epsilon\in\left[0,\infty\right)$.
\end{center}
Es gelte
\[\int_{T_1}^{T_2}dT_\epsilon L(Q\epsilon(T_\epsilon),\frac{dQ\epsilon}{dt\epsilon}(T\epsilon),T\epsilon){!\atop =}\int_{t_1}^{t_2}dt\left\{L(q(t),\dot q(t),t)+\epsilon\dfrac{df_\epsilon}{dt}(q(t),t)\right\}\]
Von dynamischer $\tau_\epsilon$-Symmetrie gefordert, mit mechanischem Eichterm $f\epsilon$.\\
Es ergibt sich eine Erhaltungsgröße $E$ der Form
\[\left.E=\sum_m\psi^m\dfrac{dL}{\partial\dot q^m}+\phi(L-\sum_m\dot q^m\dfrac{\partial L}{\partial \dot q^m})-f_{\epsilon=0}(q,t)\right|_\gamma\]


\subsubsection*{Bemerkung}

\begin{itemize}
	\item mechanische Eichtransformationen sollen in sinnvoller Weise bei dynamischer Symmetrie berücksichtigt werden $(f\epsilon(q,t)\neq 0)$
	\item Symmetrische $\tau\epsilon$ Symmetrie bedeutet:\\
	$Q_{\epsilon=0}^m(T_{\epsilon=0})$ ist Lösung der ELGl$\rightarrow $ $Q_\epsilon^m(T_\epsilon)$ ist Lösung. Sowie $Q^m_{\epsilon=0}=q^m$ und $T_{\epsilon=0}=t$
\end{itemize}

\subsubsection*{Bemerkung}

\begin{itemize}
	\item Sei Transformation $Q_{\epsilon}^m=q^m+\epsilon\psi^m(q,\dot q,t)+O(\epsilon^2)$
	\item Sei $T_\epsilon=0$ gegeben.\\
	Es folgt:
	\[\dfrac{dQ_{\epsilon}}{dT\epsilon}(T_\epsilon)=\underbrace{\dfrac{dq^m}{dt}}_{:=\dot q^m}+\epsilon\dfrac{d\psi^m}{dt}(q(t),\dot q(t),t)+O(\epsilon)\]
	Aus $\tau_\epsilon$ Symmetrie:
	\begin{align*}
		\int_{T_1}^{T_2}dT\epsilon L(Q_\epsilon(T_\epsilon),\dfrac{dQ_\epsilon}{dT\epsilon}(T_\epsilon),T_\epsilon)
		&=\int_{t_1}^{t_2}dt
		\underbrace{L\left(\left\{q^m+\epsilon\psi^m\right\}\dot q^m+\epsilon\dfrac{d\psi^m}{dt},t\right)}_{\substack{
			=L(q(t),\dot q(t),t)\\+\epsilon\left\{\sum_m \psi^m\dfrac{\partial L}{\partial q^m}+\sum_m\dfrac{d\psi^m}{dt}\dfrac{\partial L}{\partial\dot q ^m}\right\}\\+O(\epsilon^2)}}\\
		&\overset{!}{=}\int_{t_1}^{t_2}dtL(q(t),\dot q(t),t)
	\end{align*}
	mit $f_\epsilon(q,t)=0$
	\[0=\sum_m\left(\psi^m\dfrac{\partial L}{\partial q^m}+\dfrac{d\psi^m}{dt}\dfrac{\partial L}{ \partial \dot q^m}\right)=\sum_m\psi^m\left(\underbrace{\dfrac{\partial L}{\partial q^m}-\dfrac{d}{dt}\left(\dfrac{\partial L}{\partial \dot q^m}\right)}_{=0\text{für} \gamma \text{Lsg}} \right)+\dfrac{d}{dt}\sum_m\left(\psi^m\dfrac{\partial L}{\partial \dot q^m}\right)\]
	Es folgt
	\[\left.E=\sum_m\psi^m\dfrac{\partial L}{\partial\dot q^m}\right|_\gamma\]
	\item Sei $Q_\epsilon^m=q^m$, $T_\epsilon=t+\epsilon$\\
	$\rightarrow \dfrac{dQ_\epsilon^m}{dT_\epsilon}=\dfrac{dq^m}{dt}\equiv \dot q^m$
	\begin{align*}
		\int_{t_1}^{t_2}dT_\epsilon L(Q_\epsilon(T_\epsilon),\dfrac{dQ_\epsilon}{dT_\epsilon}(T_\epsilon),T_\epsilon)&=\int_{t_1}^{t_2}dt\underbrace{L(q(t),\dot q(t),t+\epsilon)}\overset{!}{=}\int_{t_1}^{t_2}dt L(q(t),\dot q(t),t)\\
		&=L(q(t),\dot q(t),t)+\epsilon\dfrac{\partial L}{\partial t}+O(\epsilon^2)
	\end{align*}
	Aus $\tau_\epsilon$-Symmetrie: $\dfrac{\partial L}{\partial t}=0$
	\begin{align*}
		\dfrac{dL}{dt}=\sum_m(\dot q^m\dfrac{\partial L}{\partial q^m}+&\underbrace{\ddot{q}^m\frac{\partial L}{\dot q^m}})+\dfrac{\partial L}{\dot q^m}=\sum_m\dot q^m\left(\frac{\partial L}{\partial q^m}-\dfrac{d}{dt}\left(\dfrac{\partial L}{\partial \dot q^m}\right)\right)+\dfrac{d}{dt}\sum_m\dot q^m\dfrac{\partial L}{\partial \dot q^m}+\dfrac{\partial L}{\partial t}\\
		&=\dfrac{d}{dt}\left(\dot q^m\dfrac{\partial L}{\partial \dot q^m}\right)-\dot q^m\dfrac{d}{dt}\left(\dfrac{\partial L}{\partial \dot q^m}\right)\\
	\end{align*}
	$=\dfrac{d}{dt}\sum_m\dot q^m\dfrac{\partial L}{\partial \dot q^m}+\dfrac{\partial L}{\partial t}$\\
	$\Leftrightarrow$ für $\gamma$ Lösung der E-L-Gl: $\dfrac{d}{dt}\left.\left(L-\sum_m\dot q^m\dfrac{\partial L}{\partial \dot q^m}\right)\right|_\gamma=\left.\dfrac{\partial L}{\partial t}\right|_\gamma$\\
	$\tau_\epsilon$-Symmetrie: $\dfrac{d}{dt}(L-\sum_m\dot q^m\dfrac{\partial L}{\partial \dot q^m})=0$ (Erhaltungsgröße)
\end{itemize}


\subsubsection*{Beispiel: der freie Massenpunkt}

\[L(x,\dot x,t)=\dfrac{m}{2}\sum_{i=1}^{3}(\dot x_i)^2\]

\paragraph*{zu Zeigen:} 10-parametrige Symmetriegruppe (eigentliche orthochrone Galilei Gruppe).

\subparagraph*{Transformationsgesetze}:

\begin{align*}
	\vec x'&=R(\underbrace{\vec{\omega}}_3)\vec x+\underbrace{\vec a}_3+\underbrace{\vec v}_3 t\\
	T&=t+\underbrace{s}_3
\end{align*}
\begin{enumerate}
	\item Translation: $\vec a\neq 0$\\
	$\tau_{trans}:$ $x_k=x_k+a_{k_0}\delta_{kk_0}$, $T=t$\\
	$\rightarrow $L-invariant unter $\tau_\epsilon$; $f,\phi=0$; $\psi_k=\delta_{kk_0}$\\
	$P_{k_0}=\sum_k\psi_k\dfrac{\partial L}{\partial \dot x_k}=\dfrac{\partial L}{\partial \dot x_{k_0}}$ ist Erhaltungsgröße\\
	$\vec P(t)=m\dot{\vec x}(t)$ Impuls ist konstant entlang Lösung.
	\item Translation in der Zeit: $s\neq 0$\\
	$\tau_{tZeit}:$ $X_k=x_k$, $T=t+s$\\
	$-E_{kin}=L-\sum_k\dot x_k\dfrac{\partial L}{\partial \dot x_k}=-\frac{m}{2}\dot{\vec x}^2$ Erhaltungsgröße.
	\item Drehungen $\vec{\omega}=c\vec e_3$, $c\in\R\backslash0$\\
	$\tau_{rot}:$ $\vec X=R(\vec \omega)\vec x$,$T=t$\\
	$\vec X=x+\overbrace{c\vec e_3}^{\equiv\vec\omega}\times\vec x+O(\epsilon^2)$\\
	$\dot{\vec X}=R(\omega)\vec{\dot x}$\\
	Somit $\dot{\vec X}^2=\dot{\vec x}^2$ $\rightarrow $ L-Invarianz mit $f,\phi=0$
	\[X_l=x_l+c\underbrace{\sum_{k=1}^{3}\epsilon_{l3k}x_k}_{:=\psi_l}+O(\epsilon^2)\]
	\[E=\sum_{k,l=1}^{3}\epsilon_{l3k}x_k\dfrac{\partial L}{\partial \dot x_l}=\vec e_3\underbrace{(\vec x\times m\dot{\vec x})}_{:=L}\]
	Sodass $\dfrac{d\vec L}{dt}=0$
	\item eigentliche Galilei Transformation: $\vec v=v\vec e$\\
	$\tau:$ $X_k=x_k+ct\vec e$, $T=t$\\
	$\dot x_k=\dot x_k+c\vec e\rightarrow f_c(x,t)\neq 0$
\end{enumerate}


\subsubsection[Wechselwirkende Systeme]{Wechselwirkende Systeme (konservativ, durch Langeragefunktion beschrieben)}
\[L(x,\dot x,t)=\sum_{i=1}^{N}\dfrac{m_{(i)}}{2}\dot{\vec x}_{(i)}^2+U(x)\]
für $\vec F_i(t)=-\vec{\nabla}_{\vec x_i}U(x(t))$\\
Damit eine eigentliche orthochrone Galilei Gruppe eine Symmetriegruppe ist muss für $U(x)$ gelten (hinreichend), \begin{itemize}
	\item $U(x)$ invariant unter $G_{10}$
	\item $U(x)$ muss translationsinvariant sein: z.B. $U(x)=\sum_{1=i<j}^{N}U_{ij}(\vec x_i-\vec x_j)$
	\item $U(x)$ muss rotationsinvariant sein: z.B. $U(x)=\sum_{i,j}U_{ij}(|\vec x_i-\vec x_j|)$
\end{itemize}

\subsection{Mechanische Systeme mit Nebenbedingunge} %2.4

\begin{itemize}
	\item holonome Nebenbedingungen $f( q(t),t)=0$, $y^2+y^2=R^2\Rightarrow f( x(t),y(t))=x^2-y^2-R^2=0$\\
	$\Rightarrow $ globale Einschränkung der Dynamik\\
	\item nicht holonome Nebenbedingungen: Involvieren Ungleichungen, Differentiale\\
	(Schränken die Teilchenbewegung auf einen kleinen Bereich der Dynamikmanigfaltigkeit ein)
\end{itemize}


\subsubsection{Holone NB und Lagrange-Gl zweiter Art} %2.4.1
Gegegben: $L(q,\dot q,t),\left\{q_i,i=1,...,3N\right\}$ mit verallgemeinerten Koordinaten $q_i$\\
\[F_l(q,t)=0; l=1,...,r\]
Die Zwangsbedingungen schränken die Teilchendynamik auf einen $f:=(3N-r)$-dinensionale Unterraum (Untermannigfaltigkeit) ein.\\
Stellen eine Matrix auf:

$\begin{pmatrix}
\dfrac{\partial F_l}{\partial q_k}
\end{pmatrix}_{r_{xk}}$ zwangsbedingungs-Matrix $F_{lk}$ mit Rang $F_{lk}=r^k\forall q\in\R^{3N}$\\
$\Rightarrow q_l(Q_1,...,Q_{3N-r},t), l=1,...,3N$\\
$\Rightarrow L(q,\dot q,t)=L\left(q( \hat Q,t),\sum_{k=1}^{3N-r}\dfrac{\partial q}{\partial Q_k}\dot Q_k+\dfrac{\partial q}{\partial t},t\right):=\tilde{L}(\hat Q,\hat{\dot Q},t)$, wobei $\hat Q\equiv(Q_1,...,Q_{3N-r})$\\
$\Rightarrow$ Euler Lagrange Gl.
\[\dfrac{d}{dt}\left(\dfrac{\partial\tilde L}{\partial\dot Q_k}\right)=\dfrac{\partial\tilde L}{\partial Q_k}; k=1,...,3N-r\]
\[\dfrac{d}{dt}\left(\dfrac{\partial \tilde L}{\partial \dot X_i}\right)=F_i(t)+Z_i(t),X( q(Q,t),t)\]
\begin{itemize}
	\item Für die kinetische Energie gilt: $\dfrac{\partial F}{\partial t}=0$
	\[\sum_j\dfrac{m_j}{2}\dot x_j\dot x_j=\sum_{ljk}\dfrac{m_{(j)}}{2}\dfrac{\partial x_j}{\partial Q_u}\dfrac{\partial x_i}{\partial Q_l}\dot Q_k\dot Q_l:=\sum_{k,l}g_{kl}(Q)\dot Q_k\dot Q_k\]
	$g_{kl}(Q)\equiv \sum_j\dfrac{m_{(j)}}{2}\dfrac{\partial x_j}{\partial Q_k}\dfrac{\partial x_j}{\partial Q_l}$ symmetrisch ($g_{kl}=g_{lk}$), positiv semidefinit
\end{itemize}

\begin{exm}
	Sphärisches Pendel
	\[L(\vec x,\dot{\vec x},t)=T-V=\dfrac{m_2}{2}\dot\vec{x}^2-(-m\vec g\vec x)=\dfrac{m}{2}\dot{\vec x}^2+m\vec g\vec x\]
	$\vec g=\begin{pmatrix}
	0\\0\\-g
	\end{pmatrix}$, $\vec x=\begin{pmatrix}
	l\sin\theta\cos\varphi\\l\sin\theta\sin\varphi\\l\cos\theta
	\end{pmatrix}\Rightarrow\dot{\vec x}=\begin{pmatrix}
	l(\dot{\theta\cos\theta\cos\varphi-\sin\theta\sin\varphi\dot{\varphi}})\\
	l(\dot{\theta}\cos\theta\sin\varphi+\sin\theta\cos\varphi\dot{\varphi})\\-l\sin\theta\dot{\theta}
	\end{pmatrix}$ 
	\[L=\dfrac{m}{2}l^2(\dot{\theta}^2+\dot \varphi^2\sin^2\theta)-mgl\cos\theta\]
	Erhaltungsgrößen:\\
	$\dfrac{\partial L}{\partial \varphi}=0$ $\Rightarrow$ $\dfrac{d}{dt}(m^2\sin^2\theta\dot{\varphi})=0$, $L_z:=,(\vec x\times\dot{\vec x})\vec e_z$\\
	$\dfrac{\partial L}{\partial t}=0$ $\Rightarrow$ $\dfrac{d}{dt}\underbrace{\left(\sum_l \dot q_l\dfrac{\partial L}{\partial \dot q_l}-L\right)}_{\substack{:=\frac{m}{2}l^2(\dot \theta^2+\dot \varphi^2\sin^2\theta)+mgl\cos\theta\\:=E}}=0$\\
	(Legendre Transformation von Lagrange zu Hamilton Funktion)
	\[E=\dfrac{m}{2}l^2\left\{\dot{\theta^2}+\dfrac{L_z^2}{m^2l^4\sin^4\theta}\right\}+mgl\cos\theta?\dfrac{m}{2}l^2\dot\theta^2+U_{eff}(\theta)\]
	\[U_{eff}(\theta):=\dfrac{L_z^2}{2ml^2\sin^4\theta}+mgl\cos\theta\]
	Fallunterscheidung
	\begin{itemize}
		\item $L_z=0\rightarrow \dot\varphi=0$ (ebenes Pendel)
		\item $L_z\neq 0\rightarrow$ $U_{eff}$ singulär bei $\theta=0,\pi$
	\end{itemize}
	$E_{min}$ bei $L_z\neq 0$, Teilchen vollführt eine Kreisbewegung $\dot\varphi=\dfrac{L_z}{ml^2\sin^2\theta_s}$\\
\end{exm}

%VL 3.11.2015
\subsubsection{Nicht holonome NB und Lagrange-Gl 1. Art}
Gegeben sei ein System mit $f$ Freiheitsgraden
\[\sum_{j=1}^{f}a_{kj}(q_j,t)dq_j+b_{(u)}(q,t)dt=0,\quad k=1,...,s\]
Rang$a_{kj}=A$ (Anzahl der Zwangsbedingungen)
\[\Rightarrow\sum_{j=1}^{f}a_{kj}(q_j,t)\dfrac{dq_i}{dt}+b_k(q,t)\]
Mit hilfe virtueller Verrückungen hat man:
\[\sum_{j=1}^{f}a_{kj}(q_j(t),t)dq_j(t)=0,\quad k=1,...,s\]
i.A. hat man $(f-s)$ unabhängige virtuelle Verrückungen $\delta q'(t),...,\delta q_{f-s}(t)$\\
Holone NB: $F_{(k)}(q,t)=0$ für $k=1,...,s$
\[dF_{(k)}=\sum_{j=1}^{f}\dfrac{\partial F_k}{\partial q_i}+\dfrac{\partial F_k}{\partial t}dt=0\]
$dF_k$ existieren nur dann wenn Integrabilitätsbedingungen gelten (Satz von Schwarz)

	\[\begin{array}{|ccccc|}
	\hline 1)&\dfrac{\partial^2 F_k}{\partial q_i\partial q_j}&\equiv\dfrac{\partial a_{kj}}{\partial q_i}&=\dfrac{\partial^2 F_k}{\partial q_j\partial q_i}&=\dfrac{\partial a_{kj}}{\partial q_j}\\
	2)&\dfrac{\partial ^2F_k}{\partial q_j \partial t}&=\dfrac{\partial b_k}{\partial q_j}&=\dfrac{\partial ^2 F_k}{\partial t \partial q_j}&=\dfrac{\partial a_{kj}}{\partial t}\\
	\hline
	\end{array}\]
Eine Zwangsbedingung $F_k$ ist holonom, wenn $dF_k$ existiert und 1) und 2) gelten, sonst ist $F_k$ nicht holonom.

\paragraph*{Das Hamilton Prinzip}(Optimierung des Wirkungsfunktionals)
\[NB:\quad \sum_{j=1}^{f}a_{kj}(q(t),t)\dot{q}(t)=0,\quad k=1,...,s\]
Stationäre Lösungen des Variationsproblems $f+s$ Euler-Lagrange-Gleichungen 1. Art.
\[\dfrac{d}{dt}\left(\dfrac{\partial L}{\partial \dot q_s}=\dfrac{\partial L}{\partial q_j}\right)+\sum_{k=1}^{s}\la_k(t)a_{kj}(q(t),t)\quad j=1,...,f\]
\[\sum_{k=1}^{s}\la_k(t)a_{kj}(q(t),t)+\sum_{k=1}^{s}b_k(q(t),t),\quad k=1,...,s\]
f Bahnkurven $q_j(t)$ und $s$ Lagrange-Multiplikatoren $\la_k(t)$\\
\\
Für holone Zwangsbedingungen:
\[a_{kj}(q,t)=\dfrac{\partial F_k}{\partial q_j}(q,t)\]
\[\tilde{L}(q,\dot q\la, t)=L(q,\dot q,t)+\sum_{k=1}^{s}\la_k F_k(q,t)\text{Effektive Lagrange Funktion}\]


\begin{theorem}
	Für zeitliche Veränderung der Energie
	\[E:=\left.\left(\sum_{j=1}^{k}\dot q_j(t)\dfrac{\partial L}{\partial \dot q_j}-L\right)\right|_\gamma\]
	ist für $\frac{\partial L}{\partial t}=0$ und holonome, Zeitunabhängige $F_k(q,t)=0$ eine Erhaltungsgröße auch für $\tilde{L}$.
\end{theorem}

\begin{proof}
	\begin{align*}
		\dfrac{\partial\tilde L}{\partial\dot q_j}&=\dfrac{\partial L}{\partial \dot q_j},\quad \dfrac{\partial\tilde L}{\partial \dot\la_k}=0\\
		\dfrac{\partial \tilde L}{\partial q_j}&=\dfrac{\partial L}{\partial q_j}+\sum_{k=1}^{s}\la_k\dfrac{\partial F_k}{\partial q_j}\\
		\dfrac{\partial \tilde L}{\partial \la_k}&=F_k\\
	\end{align*}
	\begin{align*}
		\Rightarrow&\left.\dfrac{d}{dt}\left(\dfrac{\partial L}{\partial \dot q_j}\right)\right|_\gamma=\left.\dfrac{\partial L}{\partial q_j}+\sum_{k=1}^{s}\la_k(t)\dfrac{\partial F}{\partial q_j}\right|_\gamma\\
		\Rightarrow&F_k=0\\
	\end{align*}
	\[\Rightarrow E:=\left.\sum_{j}\dot q_j-L\right|_\gamma\quad\text{Ist Erhaltungsgröße}\]
\end{proof}

\begin{exm}
	ebenes Pendel mit bewegtem Aufhängepunkt $f(t)$:\\
	\[\text{ZB:} (x-f(t))^2+y^2-l^2=F(x,y,t)=0\]
	ZB ist holonom, rheonom (da Zeitabhängig)
	\[\dfrac{\partial F}{\partial t}=-2(x-f(t))\dot f(t)\neq 0\]
	Lagrange-Gl 2. Art:
	\[\begin{array}{rl}
	x-f(t)&:=l\sin\varphi\\
	y&:=-l\cos\varphi
	\end{array}\Rightarrow
	\begin{array}{rl}
	\dot x&=\dot f+l\cos\varphi\\
	\dot y&=\dot l\dot\varphi \sin\varphi
	\end{array}\]
	\[L(\varphi,\dot{\varphi}t)=\dfrac{m}{2}\left\{l^2\dot{\varphi}^2+2l\cos\varphi\dot{\varphi f}\right\}+kgl\cos\varphi\]
	\[\dfrac{\partial \dot L}{\partial\dot{\varphi}}=ml^2\dot{\varphi}+ml\cos\varphi f\]
	\[\dfrac{\partial L}{\partial \varphi}=-ml\sin\phi\dot{\varphi}\dot f-mgl\cos\varphi\]
	Daraus folgt, dass
	\[\ddot{\varphi}+\dfrac{g}{l}\sin\varphi=-\cos\varphi\dfrac{\ddot f}{l}\]
\end{exm}


%VL 10.11.2015
\section{Lineare Schwingungen} %Kapitel 3.
Lineare Systeme: einfachste, exakt lösbare Vielteilchensysteme.\\
Betrachte: mechanisches Vielteilchefsystem (N Massenpunkte) mit holonom-skleronomen Randbedingungen.
\paragraph*{Koordinaten} $x_\mu(q_i)$ mit $mu_=1,...,3N$  und $i=1,...,f<3N$\\
Koordinatentransformation, sodass $q_i$ durch Zwangsbedingungen wegfallen.\\
\paragraph*{Potenzial}: \[T=\sum_{i=1}^{N}m_i\dot{\vec x_i}\dot{\vec x_i}=\sum_{\alpha,\beta=1}^{f}\dot q_\alpha g_{\alpha\beta}(q)\dot q_\beta\]
mit $g_{\alpha\beta}=\sum_{i=1}^{N}m_i\frac{\partial \vec x_i}{\partial q_\alpha}\dfrac{\partial\vec x_i}{\partial\alpha\beta}>0$
\paragraph*{Lagrange Gleichung}:
\[L(q,\dot q)=\sum_{\alpha,\beta=1}^{N}\dot q_\alpha g_{\alpha\beta}(q)\dot q_\beta-U(q)\]
Wir setzen auch voraus:
\begin{itemize}
	\item Existenz einer Gleichgewichtskonfiguration
	\item kleine Auslenkung um diese Gleichgewichtskonfiguration
\end{itemize}


\subsubsection*{Anwendung}
\begin{itemize}
	\item Beschreibung von Festkörpern
	\item Elektromagnetisches Strahlungsfeld
	\item Systeme ungekoppelter Harmonischer Oszillatoren
\end{itemize}
Beschreibt z.B. nicht: Gase, Schmelzende Körper,...
\subsection{Lineare Schwingungen um Gleichgewichtskonfigurationen}

\begin{defn}
	In \emph{Gleichgewichtskonfiguration} gilt:\\
	$\{q_{\alpha(0)}(t)$ ist Zeitungabhängig, $\alpha=1,...,f\}$\\
	\\
	Äquivalent dazu ist $\dot q_{\alpha(0)}(t)=\ddot{q}_{\alpha(0)}=\dddot{q}_{\al(0)=...=0}$\\
\end{defn}
Es gilt für generalisierte Kräfte, dass $\left.\frac{\partial U}{\partial q_\al}\right|_{q_{(0)}}=0$

\begin{proof}[Beweisidee]
	Euler-Lagrange und Bedingugngen für Gleichgewicht.
\begin{align*}
	\frac{\partial L}{\partial \dot q_\al}&=\frac{1}{2}\sum_{\beta=1}^{f}g_{\al\beta}(q)\dot q_\beta+\frac{1}{2}\sum_{\beta=1}^{f}\dot q_\beta g_{\beta\al}(q)&\text{da }g_{\al\beta}=g_{\beta \al}\\
	&=\sum_{\beta=1}^{f}g_{\al\beta}(q)\dot q_\beta\\
	\rightarrow \text{E-L:}&\underbrace{\dfrac{d}{dt}\dfrac{\partial L}{\partial\dot q_\al}}_{=0}=\dfrac{\partial L}{\partial q_\al}=-\frac{\partial U}{\partial q_\al}\\
	\dfrac{\partial L}{\partial q_\al}&=-\frac{\partial U}{\partial q_\al}+\frac{1}{2}\sum_{\beta,\gamma=1}^{f}\dot q_\beta\frac{\partial q_{\beta\gamma}}{\partial q_\al}\\
	&=-\frac{\partial U}{\partial q_\al}
\end{align*}
Für kleine Auslenkungen nähern wir über Taylor-Polynome:\\
$q_\al(t)=q_{\al(0)}+\eta_\al(t)$, $\eta_\al(t)$ klein
\[L=\frac{1}{2}\dot{\eta}^T\underbrace{g(q_{(0)}}_{:=K})\dot{\eta}-U(q_{(0)}-\frac{1}{2}\sum_{\al,\beta=1}^{f})\nu_\alpha\underbrace{\left.\frac{\partial^2U}{\partial q_\al\partial q_\beta}\right|_{q_{(0)}}}_{:=K}\eta\beta+O^3(\eta,\dot{\eta})\]
\[L(\eta,\dot{\eta})=\dfrac{1}{2}\dot{\eta}^Tq(q_{(0)})\dot{\eta}-\dfrac{1}{2}\eta^TK\eta\]
$M$ Massenmatrix, symmetrisch positiv definit\\
$K$ Kopplungsmatrix, symmetrisch positiv definit

\end{proof}
\subsection{Dynamik kleiner Schwingungen -- Normalkoordinaten}
\[L(\eta,\dot\eta)=\dfrac{1}{2}\dot\eta M\dot \eta-\frac{1}{2}\eta^TK\eta\]
mit $M>0,M=M^t$ und $K>0,K=K^T$, sodass
\begin{align*}
	\frac{\partial L}{\partial \dot\eta_\al}&=\sum_{\beta=1}^{f}M_{\al,\beta}\dot\eta_\beta\\
	\frac{\partial L}{\partial \eta_\al}&=-\sum_{\beta=1}^{f}K_{\al,\beta}\eta_\beta\\
	\frac{d}{dt}\frac{\partial L}{\partial \dot\eta_\al}&=\sum_{\beta=1}^{f}M_{\al,\beta}\ddot\eta_\beta=-\sum_{\beta=1}^{f}K_{\al,\beta}\eta_\beta=\frac{\partial L}{\partial \eta_\al}\\
\end{align*}
mit Orthogonalmatrix $O$, $Q=O^TM^{\frac{1}{2}}\mu$, $M^{-\frac{1}{2}}KM^{-\frac{1}{2}}=OKO^T$\\
$M^{1/2}_{ij}$ Verändert die Matrix indem wir $M$ diagonalisieren, die Wurzeln der Diagonalelemente Zeihen und danach die Trasnformationsmatrizen wieder einrechnen. Sei $M=SAS^T$ und $B_{ii}=\sqrt{A_{ii}}$, dann def. $M^{1/2}=SBS^T$.
\[L(Q,\dot Q)=\frac{1}{2}\dot Q^T\dot Q-\frac{1}{2}Q^TKQ=\frac{1}{2}\sum_{i=1}^{f}\dot Q_i\dot Q_i-\frac{1}{2}\sum_{i=1}^{f}Q_iK_iQ_i\]


\subparagraph*{Beweisidee}
\[L(\eta,\dot{\eta})=\frac{1}{2}(M^{\frac{1}{2}}\dot\eta)^T(M^\frac{1}{2}\eta)-\frac{1}{2}\eta^TK\eta=\frac{1}{2}\dot{\ol{Q}}^T\dot{\ol{Q}}-\frac{1}{2}\dot{\ol{Q}}^T\ol{Q}-\dfrac{1}{2}\ol{Q}^TM^{-\frac{1}{2}}KM^{-\frac{1}{2}}\ol Q=\frac{1}{2}\dot Q^T\dot Q-\dfrac{1}{2}Q^TKQ\]
da $Q=O^TM^{\frac{1}{2}}\eta$, $O^TO=OO^T=1$

\subparagraph*{Lösungen}
\begin{itemize}
	\item $K_i>0$: $Q_i(t)=Q_i(0)\cos(\sqrt{K_i}t)+\frac{\dot Q_i(0)}{\sqrt k_i}\sin(\sqrt{K_i}t)$
	\item $K_i=0$: $Q_i(t)=Q_i(0)+\dot Q_i(0)t$
	\item $K_i<0$: $Q_i(t)=Q_i(0)\cosh(\sqrt{K_i}t)+\frac{\dot Q_i(0)}{\sqrt k_i}\sinh(\sqrt{K_i}t)$ 
\end{itemize}


\subsubsection{Modell eines lineare dreiatomigen Moleküls}
\begin{figure}[hc]
	\begin{center}
		\begin{tikzpicture}[scale=0.7]
		\draw[decorate,decoration=snake](-3,0)--(3,0);
		\draw[fill=black](0,0)circle(0.4);
		\node at(0,-0.3)[below]{$\vec x_2$}; 
		\node at(0,0.3)[above]{$M$};
		\draw[fill=black](-3,0)circle(0.2) node[below]{$\vec x_1$} node[above]{$m$};
		\draw[fill=black](3,0)circle(0.2) 
		node[below]{$\vec x_3$} node[above]{$m$};
		\node at(-1.5,0)[below]{$k$};
		\node at(-1.5,0)[above]{$l_0$};
		\node at(1.5,0)[below]{$k$};
		\node at(1.5,0)[above]{$l_0$};
	\end{tikzpicture}
	\end{center}
	\caption{Lineares dreatomiges Modekül}
	\label{fig:lin3at}
\end{figure}

\begin{align*}
	M&=\begin{pmatrix}
	m&0&0\\0&M&0\\0&0&m
	\end{pmatrix}&K&=\begin{pmatrix}
	k&-k&0\\-k&2k&-k\\0&-k&k
	\end{pmatrix}
\end{align*}

\[L(\eta\dot{\eta})=\frac{1}{2}m\dot x_1^2+\frac{1}{2}m\dot x_3^2+\frac{1}{2}M\dot x_2^2\underbrace{-\frac{1}{2}k(x_2-x_1-l_0)^2-\frac{1}{2}k(x_3-x_2-l_0)^2}_{:=-U(x)}\]
Es ergibt sich zusammengefasst:
%VL 17.11.2015
\[L(\eta,\dot \eta)=\frac{1}{2}\dot{\eta}^TM\dot{\eta}-\dfrac{1}{2}\eta^TK\eta\]
Wir bestimmen nu die Massen-Kopplungs-Matrix
\[M^{-\frac{1}{2}}KM^{-\frac{1}{2}}=\begin{pmatrix}
\frac{k}{m}&-\frac{k}{\sqrt{mM}}&0\\-\frac{k}{\sqrt{mM}}&\frac{2k}{M}&-\frac{k}{\sqrt{mM}}\\0&-\frac{k}{\sqrt{mM}}&\frac{k}{m}
\end{pmatrix}\]
Wir Bestimmen eine Matrix $O$ um $M^{-\frac{1}{2}}KM^{-\frac{1}{2}}$ zu Diagonalisieren, indem wir die Eigenwerte bestimmen:
\[M^{-\frac{1}{2}}O=\begin{pmatrix}
\frac{1}{2m}&\frac{1}{\sqrt{2m+M}}&\frac{1}{\sqrt{2m(1+\frac{2m}{M})}}\\
0&\frac{1}{\sqrt{2m+M}}&-\frac{2m/M}{\sqrt{2m(1+\frac{2m}{M})}}\\
-\frac{1}{\sqrt{2m}}&\frac{1}{\sqrt{2m+M}}&\frac{1}{\sqrt{2m(1+\frac{2m}{M})}}
\end{pmatrix}\]
Jede Spalte stellt dabei eine Mögliche Schwingungsmode (Energieverteilung) dar.

\paragraph*{Bemerkung}
\begin{itemize}
	\item Transaktionsmoden im $\R^3$: 3 Freiheitsgrade
	\item Grenzfall: $k\rightarrow \infty$ Relativbewegungen unmöglich
		$\rightarrow $ Starrer Körper; 6 Freiheitsgrade
\end{itemize}

\subsubsection*{Zusammenfassung}
Lagrange Gleichung: $L(\eta,\dot{\eta})=\frac{1}{2}\dot{\eta}^TM\dot\eta-\dfrac{1}{2}\eta^TK\eta$\\
Mit Massenmatrix M, Kopplungsmatrix K (symmetrisch, positiv definit).\\
\[M\ddot{\eta}=-K\eta\]
Die allgemeinen Koordinaten unseres System berechnen sich als $Q=O^TM^{1/2}\eta$\\
Wobei $O$ definiert ist über $M^{-1/2}KM^{-1/2}=O\kappa O^t$\\
Es folgt, dass $\dot Q=O^TM^{1/2}\dot{\eta}=\frac{1}{2}\dot Q^T\dot Q-\frac{1}{2}Q^T\kappa Q$

\newpage
\subsection{Dynamische Vielteilchensysteme: Der schwingende Ring} %3.3
\begin{figure}[h]
	\begin{center}
	\begin{tikzpicture}
		\draw (0,0) circle (2);
		\foreach \t in {0,...,16} {
		\coordinate (a) at ({2*sin(\t*360/16)},{2*cos(\t*360/16)});
		\draw[decorate,decoration=snake](a)--({2*sin((\t+1)*360/16)},{2*cos((\t+1)*360/16)});
		\draw[fill=black](a) circle(0.07);
		\draw[decorate, decoration=snake] (a)--({0.7*sin(\t*360/16)},{0.7*cos(\t*360/16)});
		};
	\end{tikzpicture}
	\end{center}
	\caption{Schwingender Ring}
	\label{fig:ring}
\end{figure}
$a=x"i a_0-x"{i-1}_0$: Abstand der Gleichgewichtspositionen\\
(mit $i=1,...,N$ und $N+1\equiv1$)\\
$x"i-x_0"i=\eta_i$\\
$K$: ???


\subsubsection{ }
\begin{align*}
	L(\eta,\dot \eta)&=\frac{1}{2}\sum_{i=1}^{N}(\dot{\eta}_i)^2m-\frac{1}{2}\eta^T\kappa\eta\\
	&=\frac{1}{2}\dot{\eta}^TM\eta-\sum_{i=1}^{N+1}(\eta"i\eta"{i-1})^2k
\end{align*}
\begin{align*}
	M&=m\cdot \mathbb 1&\kappa&=\begin{pmatrix}
	2k+K&-k&&&-k\\
	-k&2k+K&-k&&\\
	&-k&2k+K&-k&\\
	&&&&\\
	-k&&&&
	\end{pmatrix}
\end{align*}

\paragraph*{Modenbestimmung}$r=1,...,N$\\
\begin{itemize}
	\item $\eta_r=\Re(C_re^{-i\omega_r t})$
	\item $\omega_r$: Eigenwerte
	\item $(-M\omega_r^2+K)l_r=0$ Eigenwertproblem
\end{itemize}


\paragraph*{Ansatz} $l_r^l=e^{il\Phi_r}$, da $C"{N+1}=C_r"1$
\begin{align*}
	\Rightarrow e^{i(N+1)\Phi_r}&=e^{i\Phi_r}\\
	N\Phi_r&=2\pi r\\
	\Phi_r&=\frac{2\pi}{N}r
\end{align*}
\begin{align*}
	0=\left[-\omega_r^2+\frac{k}{m}(2-e^{i\Phi_r}-e^{i\Phi_r})+\frac{K}{m}\right]C_r"l
	\intertext{Wobei gilt:}
	2-e^{i\Phi_r}-e^{i\Phi_r}&=2(1-\cos\Phi_r)\\
	&=4\sin^2\left(\frac{\Phi_r}{2}\right)
	\intertext{Sodass}
	0\leq\omega_r&=\sqrt{\frac{K}{m}+4\frac{k}{M}\sin^2\left(\frac{\Phi_r}{2}\right)}\\
	&=\sqrt{\frac{K}{m}+4\frac{k}{M}\sin^2\left(\frac{\pi r}{N}\right)}
\end{align*}
mit $r=1,...,N$ und $C_r"l=e^{i\frac{2\pi r}{N}l}$.\\
\\
Wir betrachten nun $\sin^2\left(\frac{\pi r}{N}\right)$. FÜr $\frac{\pi r}{N}=0$ oder $=\frac{\pi}{2}$ gibt es keine Entartung uns somit nur eine Lösung, sonst $(\frac{\pi r}{N}\in(0,\pi)]\backslash \{\frac{\pi}{2}\})$ gibt es 2 Lösungen.
\begin{align*}
\eta"i(t)&=\sum_{i=1}^{N}|A_r|Re\{e^{-i\omega_rt}e^{i\phi_r}e^{i\frac{2\pi r}{N}l}\}\\
&=\cos\{\omega_r t-\frac{2\pi r}{N}l-\phi_r\}
\end{align*}
\begin{bem}
	\begin{itemize}
		\item Mode $r$: $\omega_r=\sqrt{\frac{K}{m}}$\\
		$\rightarrow $ $\eta_{r=N}"l(t)$ ist eine Translation entlang des Rings
		\item Wellenlänge und Phasengeschwindigkeit\\
		$\psi_r=\omega_rt-\frac{2\pi r}{N}l-\phi_r$. Für $x_o"1=la$ gilt:
		\begin{align*}
		\psi_r&=\omega_rt-\frac{2\pi r}{Na}(la)-\phi_r\\
		0=d\psi_r&=\omega_r~dt-\frac{2\pi r}{N}+dx_0"l\\
		v_{ph}&=\frac{dx_0"l}{dt}=\underbrace{\frac{\omega_r}{2\pi r}}_{\nu_r}\underbrace{Na}_{\la}
		\end{align*}
		Im Langwellenlimes $\frac{\lambda_r}{a}=\frac{N}{r}\gg 1$ 
		(also $\sin \frac{\Pi r}{N}\approx \frac{\Pi r}{N}$) ergibt sich somit:
		\begin{align*}
		v_{Ph}&\longrightarrow&\sqrt{\frac{K}{m}\left(\frac{Na}{2\Pi r}\right)^2
		+\frac{4k}{m}\left(\frac{\Pi r}{N} \frac{Na}{2\Pi r}\right)^2}\\
		&=&\sqrt{\frac{K}{m}\left(\frac{Na}{2\Pi r}\right)^2+\frac{k a^2}{m}}
		\end{align*}
		Für $K=0$ erhalten wir $\sqrt{\frac{k a^2}{m}}$, unabhängig von der Mode. 
		Für $K\neq 0,\ r=0$ (Translationsmode) gilt hingegen $v_{Ph}=\infty$, was nicht unphysikalisch ist, 
		da $v_{Ph}$ keinen Transport einer physikalischen Größe charakterisiert.
		\item Dispersionsrelation: Lösung = Dispersionsrelation+Moden
		
	\end{itemize}
\end{bem}

%TODO VL 17.11.
%VL 19.11.2015
\subsubsection{Der Kontinuierliche Grenzfall}
Für Gleichgewichtspositionen $x_n$ und ausgelenkte Position $x'_n$ \\gilt $x'_n(t)=x_n+\eta_n(t)$:

\paragraph*{Ausgangspunkt}
	\[m\ddot{\eta}_l(t)+k(2\eta_l(t)-\eta_{l-1}(t))+K\eta_l=0\]
$\Leftrightarrow M\ddot\eta+K\eta=0$.\\
\paragraph*{Kontinuumlimes}beschreibt den Grenzfall $a\rightarrow 0,N\rightarrow \infty$,sodass $Na:=L=konst$ im Fall $(\la\gg a)$: Langwellengrenzfall.\\
\begin{align*}
	\eta_l(t):=u(x_l,t)&&\text{sodass} &&\ddot\eta_l(t)&=\frac{\partial^2u}{\partial t^2}(x,t)\\
\end{align*}
\begin{align*}
	2\eta_l-\eta_{l-1}(t)-\eta_{l+1}&=2u(x_l,t)-u(x_{l-1},t)-u(x_{l+1},t)\\
	\text{Taylor Entwickelung: }&=2u(x_l,t)-\big(-u(x_l,t)-a\frac{\partial u}{\partial x}(x_l,t)+\frac{a^2}{2}\frac{\partial u^2}{\partial x^2}(x_l,t)+O(a^3)\big)\\
	&\quad -\big(u(x_l,t)+a\dfrac{\partial u}{\partial x}(x_l,t)+\frac{a^2}{2}\frac{\partial^2}{\partial x^2}(x_l,t)+O(a^3)\big)\\
	&=-a^2\frac{\partial^2 u}{\partial x^2}(x_l,t)
	\intertext{Einsetzen in die Dgl}
	\frac{1}{\frac{ka^2}{m}}\frac{\partial^2 u}{\partial t^2}(x,t)+\frac{K}{ka^2}u(x,t)=0
\end{align*}
Grenzwertbetrachtungen:
\[\begin{array}{lr}
v:=\lim\limits_{\substack{a\rightarrow 0\\m\rightarrow 0\\k\rightarrow 0}}\sqrt{\frac{ka^2}{m}}&\kappa^2:=\lim\limits_{\substack{q\rightarrow 0\\K,k\rightarrow 0}}\left(\frac{K}{ka^2}\right)
\end{array}\]
Wir erhalten die Klein-Gordon Gleichung
\[\left(\frac{1}{v^2}-\dfrac{\partial^2}{\partial x^2}+\kappa^2\right)u(x,t)\]

\subsubsection*{Energie:}
\[E=\sum_{l=1}^{N}\left(\frac{m}{2}\big(\dot{\eta}_l(t)\big)^2+\frac{k}{2}\big(\eta_l(t)-\eta_{l-1}(t)\big)^2+\frac{K}{2}\big(\eta_l(t)\big)^2\right)\]
Da $N\rightarrow \infty$ 
\begin{align*}
	E&=\int_{1}^{N}d\overbrace{la}^{:=x}\left\{\frac{m}{2a}\left(\frac{\partial u}{\partial t}\right)^2+\frac{ka}{2}\left(\frac{\partial u}{\partial x}\right)^2+\frac{K}{2a}u^2\right\}\\
	\intertext{wir definieren $\varrho=\lim\limits_{\substack{a\rightarrow \\m\rightarrow 0}}\frac{m}{a}$}
	&=\frac{1}{2}\int_{0}^{L}dx\left\{\varrho\left(\frac{\partial u}{\partial t}\right)^2+\varrho v^2\left(\frac{\partial u}{\partial x}\right)^2+\kappa^2\varrho v^2u^2\right\}
\end{align*}
Für den diskreten Ring gilt also
\[\eta_l(t)=\sum_{r=-\frac{N-1}{2}}^{\frac{N-1}{2}}A_re^{-i(\omega_Rt-\frac{2\pi}{N}lr)}\]
mit $A_r\in\C$ und $\omega_r=\sqrt{\frac{K}{m}+\frac{4k}{m}\sin^2\left(\frac{\pi r}{N}\right)}$ für $r=1,...,N$ und $L\rightarrow \infty$.\\

\subparagraph*{Kontinuumslimes}
\begin{align*}
	u(x,t)&=\int_{-\frac{N-1}{2}}^{\frac{N-1}{2}}d\left(r\frac{2\pi}{Na}\right)\underbrace{\left(\frac{A_rNa}{2\pi}\right)}_{:=A(k)}e^{-i\omega_rt-r\frac{2\pi}{Na}al}+kompl. Konj\\
	&=\int_{-\infty}^{\infty}dk\quad A(k)e^{-i\omega(k)t}e^{ikx}+kompl. Konj
\end{align*}


\subsubsection*{Einige Eigenschaften der Fourier-Transformation (FT)}
\begin{itemize}
	\item Definition der FT:
		\[\tilde\varphi:=\frac{1}{\sqrt{2\pi}}\int_{-\infty}^{\infty}dx e^{ikx}\varphi(x)\]
	\item Umkehrtransformation
		\[\varphi(x)\lim\limits_{\epsilon\rightarrow 0_+}\frac{1}{\sqrt{2\pi}}\int_{-\infty}^{\infty}dke^{ikx}e^{-\epsilon|k|}\tilde\varphi(k)\]
		%VL 24.11.2015
		\begin{proof}
			\begin{align*}
				\varphi(x)&=\lim\limits_{\epsilon\rightarrow 0^+}\frac{1}{2\pi}\int_{-\infty}^{\infty}dk\int_{-\infty}^{\infty}dx'e^{ikx}e^{-\epsilon|k|}e^{-ikx'}\varphi (x')\\
				&=\lim\limits_{\epsilon->0^+}\int_{-\infty}^{\infty}dx'\underbrace{\left\{\frac{1}{2\pi}\int_{-\infty}^{\infty}dke^{ik(x-x')}e^{-\epsilon|k|}\right\}}_{\substack{:=\frac{1}{\pi}\frac{\epsilon}{(x-x')^2+\epsilon^2}\\\text{(Lorentz-Kurve)}}}\varphi(x')\\
				&=\lim\limits_{\epsilon\rightarrow 0^+}\int_{-\infty}^{\infty}dx'\underbrace{\frac{1}{\pi}\frac{\epsilon\varphi(x')}{(x-x')^2+\epsilon^2}}_{\substack{\text{Distributionen}\\\text{(uneigentliche}\\\text{Funktionen)}}}
			\end{align*}
		\end{proof}
		\begin{figure}[h]
			\begin{center}
				\begin{tikzpicture}[scale=0.8]
				\draw[thick,->](-0.3,0)--(3,0)node[above]{$x'$};
				\draw[thick,->](0,-0.3)--(0,2);
				\draw(1,0)node[below]{x}--(1,2.1)node[above]{f(x')};
			\end{tikzpicture}
			\end{center}
			\caption{$\delta$-Funktion}
			\label{fig:delta}
		\end{figure}
	\item Faltungstheorem
		\[(\varphi*\psi)(x):=\frac{1}{\sqrt{2\pi}}\int_{-\infty}^{\infty}dx'\varphi(x-x')\psi(x')\]
		(Faltung, engl. convolution), $kx=const\Leftrightarrow k=\frac{const}{x}$(komplementäre Variablen)
	\begin{center}
		\fbox{$\Leftrightarrow\varphi*\psi=\tilde\varphi\tilde\psi$}
	\end{center}
	\item Parseval'sche Gleichung
		\[\int_{-\infty}^{\infty}dx|\varphi(x)|^2=\int_{-\infty}^{\infty}dk|\varphi(k)|^2\]
\end{itemize}


\subsubsection*{Elementare Eigenschaften von stetigen Funktionalen un Distributionen}
Gegeben sei ein stetiges,lineares Funktional über den Funktionenraum \textfrak{F}.\\
$l:\textfrak{F}\rightarrow \C$, $l(c1\varphi_1+c_2\varphi_2)=c_1l(\varphi_1)+c_2l(\varphi_2)$ (Linearität) für $c_1,c_2\in\C$ und $\varphi_1,\varphi_2\in\textfrak{F}$.\\
Funktionen $\varphi$ werden auch Testfunktionen genannt
\begin{itemize}
	\item \textfrak{D}: Raum komplexer, unendlich oft differenzierbarer Funktionen mit kompaktem Träger
	\item \textfrak{C}: Raum unendlich oft differenzierbarer komplexer Funktionen die im Unendlichen stärker abnehmen als jede noch so hohe Potenz (streben gegen Null)
\end{itemize}
\begin{itemize}
	\item Um Stetigkeit zu garantierne muss \textfrak{F} zuem topologischen Raum mit Hilfe der Funktionennorm $\norm{\varphi}$ gemacht werden.
	\item Beispiele: stetiges Funktional\\
	\begin{itemize}
		\item $l(\varphi):=\int_{-\infty}^{\infty}f(x)\varphi(x)dx$ für $f(x)$ stetig.\\
		$l$ heißt auch regulär.
		\item $l(\varphi):=\varphi(X=0)$ Dirac-Funktional\\
		$f(x)$ ist dann die Dirac-(Delta)-Funktion $f(x):=\delta(x)$
		\item Zu jedem stetigen linearen FUnktional über \textfrak{D} oder \textfrak{C} existiert eine FOlge $f_n$ stetiger Funktional-Kerne mit
		\[l(\varphi)=\lim\limits_{n\rightarrow \infty}\int_{-\infty}^{\infty}dxf_n(x)\varphi(x):=\int_{-\infty}^{\infty}f(x)\varphi(x)dx\quad\text{für bel $\varphi\in\textfrak{D}$ oder $\textfrak{C}$}\]
		$\Rightarrow$ $f$ nennt man Distribution.
	\end{itemize}
\end{itemize}

\paragraph*{Eigenschaften von $\delta(x)$}
\begin{itemize}
	\item $\int_{-\infty}^{\infty}dx\delta(x-x_0)\varphi(x)=\varphi(x_0)$
	\item Es gilt: Im Limes $n\rightarrow \infty$
\begin{itemize}
	\item $f_n(c):=ne^{-n^2\pi x^2}\rightarrow \delta(x)$, denn
	\begin{align*}
	\lim\limits_{n\rightarrow \infty}\int_{-\infty}^{\infty}f_n(x)\varphi(x)dx&=\lim\limits_{n\rightarrow \infty}\frac{n}{n\sqrt \pi}\int_{-\infty}^{\infty}due^{-u^2}\varphi(u\frac{n}{n\sqrt \pi})\\
	&=\frac{1}{\sqrt \pi}\varphi(0)\int_{-\infty}^{\infty}due^{-u^2}=\frac{1}{\sqrt{\pi}}\varphi(0)\sqrt{\pi}=\varphi(0)
	\end{align*}
	\item $f_(x):=\frac{n}{\pi}\left(\frac{\sin(nx)}{xn}\right)^2\rightarrow \delta(x)$
	\item $f_n(x):=\frac{1}{\pi}\frac{\frac{1}{n}}{x^2+\left(\frac{1}{n}\right)^2}\rightarrow \delta(x)$
	\item $f_n(x):=\frac{1}{\pi}\frac{\sin(nx)}{x}\rightarrow \delta x$
\end{itemize}
	\item Definition $l'(\varphi):=-l(\varphi')$ für alle Testfunktionen aus \textfrak{F}
	\[l'=\int_{-\infty}^{\infty}dxf'(x)\varphi(x)=\left[f(x)\varphi(x)\right]_{-\infty}^{\infty}-\int_{-\infty}^{\infty}dxf(x)\varphi'(x)\]
	$\Rightarrow$ Jede Distribution auf \textfrak{C} lässt sich als Distributionsableitung endlicher Ordnung einer stetigen Fkt. darstellen.
	\item Beispiel: sei $\alpha(x):=\begin{cases}
		0&x<0\\
		x&x\ge0
		\end{cases}$\\
		Im Distributionssinn gilt:
		\[\alpha'(x)=\theta(x)=\begin{cases}
		0&x<0\\1&x>0
		\end{cases}\]
	\item Es gilt
	\begin{itemize}
		\item $x\delta(x)=0$
		\item $\delta(-x)=\delta(x)$
		\item $\delta(g(x))=\sum_l\frac{1}{|g(x)|\delta(x-x_l)}$, mit stetigem $g$,$g(x_l)=0$ und $g'(x_l)\neq 0$.
		\item $\delta(ax)=\frac{1}{|a|}\delta(x)$
		\item $\tilde{\delta k}=\frac{1}{\sqrt{2\pi}}$
		\item $\delta(x-x_0)=\frac{1}{2\pi}\int_{-\infty}^{\infty}dk\quad e^{ik(x-x_0)}\cdot 1$ (Fourier-Transformation über 1)
	\end{itemize}
\end{itemize}

\paragraph*{Bestimmung der Amplituden $A(k)$ aus den Anfangsbedingungen}
\begin{align*}
	U(x,0)&=\int_{-\infty}^{\infty}dk(A(k)e^{ikx}+A^*(k)e^{ikx}):=F(x)\\
	\frac{\partial U}{\partial t}(x,0)&=\int_{-\infty}^{\infty}dx\left\{A(k)(-i\omega(k))^{ikx}\right\}+A^*(k)i\omega(k)e^{-ikx}:=G(x)\\
	A(k)&=\frac{i\omega(k)\tilde F(k)-\tilde G(k)}{2i\omega(k\sqrt{2\pi})}
\end{align*}

%VL 26.11.2015? oder so
\paragraph*{Bestimmung der Lösungen für Randbedingungen}z.B. (W) für $x\in[0,L]$ mit $0=u(x=0,t)$ und $u(x=L,t)=0$ vorgegeben für alle $t$.\\
\[\left(\frac{1}{v^2}\frac{\partial ^2}{\partial t^2}-\frac{\partial^2}{\partial x^2}+\kappa^2\right)u(x,t)=0\tag{W}\]
zusätzliche Anfangsbedingungen:
\[u(x,0),\frac{\partial u}{\partial t}(x,0)\text{ für }x\in[0,L]\]
Fouriertransformation:
\[u(x,t)=\int_{-\infty}^{\infty}dk\underset{\in\C}{A(k)}e^{ikx}e^{-i\omega(k)t}+k.k.\]
Randbedingung:
\begin{align*}
	0=u(x=0,t)&=\int_{-\infty}^{\infty}dkA(k)e^{-i\omega(k)t}+k.k.\\
	0=u(x=L,t)&=\int_{-\infty}^{\infty}dkA(k)e^{-i\omega(k)t}e^{ikL}+k.k.\\
	0&=\int_{-\infty}^{\infty}dkA(k)e^{-i\omega(k)t}\underbrace{(1+e^{ikL})}_{=0}+k.k.
\end{align*}
$1+e^{ikl}=0\Rightarrow kL=n\pi,n\in\Z\Rightarrow k_n=\frac{\pi}{L}n$\\
$\Rightarrow A(k)=A_n\delta(k-\frac{pi}{L}n)$
\[\Rightarrow\quad u(x,t)=\sum_{n=-\infty}^{\infty}A_ne^{-i\omega(k_n)t+ik_nx}+k.k.\quad\text{mit $k_n=\frac{n\pi}{L}$}\]
zusätzlich:
\begin{align*}
	u(x=0,L)&=0\\
	&=\sum_{n=1}^{\infty}(A_n+A_{-n})e^{i\omega(k_n)t}+A_0ee^{-i\omega(k=0)t}+k.k.\\
\end{align*}

$\Rightarrow $ $A_n+A_{-n}=0$ und $A_0=0$.

\begin{align*}
	u(x,t)&=\sum_{n=\infty}^{\infty}A_ne^{-i\omega (k_n)t}e^{i\frac{k_n\pi}{L}x}+\sum_{n=-\infty}^{\infty}A_n^*e^{i\omega(k_n)t}e^{-i\frac{n\pi}{L}x}\\
	&=\sum_{n=1}^{\infty}\underbrace{\left(A_ne^{i\frac{n\pi}{L}x}+A_{-n}e^{i\frac{n\pi}{L}x}\right)}_{=A_n\cdot2 i\sin\left(\frac{n\pi}{L}x\right)-A_n}e^{-i\omega(k_n)t}+k.k\\
	&=\sum_{n=1}^{\infty}2|A_n|\sin\left(\frac{n\pi}{L}x\right)\cdot 2\Re\{e^{-i\omega(k_n)t+i\phi_n}i\}\\
	&=\sum_{n=1}^{\infty}4|A_n|\sin\left(\frac{\pi n}{L}x\right)\sin(\omega(k_n)t-\phi_n)
	\intertext{$\sin\left(\frac{\pi n}{L}x\right)$ bilden dabie ein vollständiges UFnktionensystem für stetige Funktionen $f$ im Intervall $[0,L)$ mit $f(x=0)=f(x=L)=0$ (sogar orthogonals Funktionensystem)}
\end{align*}
Es gilt
\[\int_{0}^{L}\left(\sqrt{\frac{2}{L}}\sin\left(\frac{n\pi}{L}x\right)\right)\left(\sqrt{\frac{2}{L}}\sin\left(\frac{m\pi}{L}x\right)\right)=\delta_{nm}\]
d.h. $v=\sum_ia_ie_i$, $w=\sum_i b_ie_i$, $(v,w)=\sum_ia_ib_i$. Somit
\begin{align*}
	\int_{0}^{L}\left(\sqrt{\frac{2}{L}}\sin\left(\frac{m\pi}{L}x\right)\right)u(x,0)=\sum_{n=1}^{\infty}4|A_n|\delta_{nm}=-4|A_m|\sin\varphi_m\\
	\dif{u}{t}(x,0)=\sum_{n=1}^{\infty}4|A_n|\sin\left(\frac{n\pi}{L}x\right)\omega(k_n)\cos(\omega(k_n)t-\varphi_n)
\end{align*}
???, das heißt Linear-Kombination von allgemeiner homogene und partikulärer Lösung.\\
$\rightarrow $???
$\rightarrow $ ???


\subsection[Erzwungene Schwingungen]{Erzwungene Schwingungen und Greensche Funktionen}
\begin{enumerate}
	\item Transformation auf Normalkoordinaten $\rightarrow $ ungekoppelte erzwungenen Schwingungen
	\item $\ddot Q(t)+\omega^2Q(t)=\textfrak{F}(t)$ linear \underline{inhomogen}
	\item Allgemeine Lösung
	\[Q(t)=\underbrace{Aq_1(t)+Bq_2(t)}_{\substack{\text{Lösung der}\\\text{homogenen Gleichung}}}+\underbrace{Q_p(t)}_{\substack{\text{partikuläre}\\\text{Lösung}}}\]
	\item Konstruktion von $Q_p(t)$:\\
	Idee: $\textfrak{F}(t)=\int_{-\infty}^{\infty}dt'\delta(t-t')\textfrak{F}(t')$
	\[\textfrak{L}G(t,t')=\frac{\partial^2}{\partial t^2}G(t,t')+\omega^2G(t,t')=\delta(t-t')\]
	\[Q_p(t)=\int_{-\infty}^{\infty}dt'G(t,t')\textfrak{F}(t')\text{ denn }\textfrak{L}Q_p(t)=\int_{-\infty}^{\infty}\textfrak{L}G(t,t')\textfrak{F(t')}=\textfrak{F}(t')\]
\end{enumerate}

\subsubsection{Lösung von DGL mit Hilfe von Integraltransformationen (Fourier, Laplace)}
$y''(x)-2y'(x)+y(x)=x^3e^x$; $y(0)=y'(0)=0$; $y(x)=f=f(x)$
Laplace-Transformation $L(y)=F(p)=\int_{0}^{\infty}dx f(x)e^{-px}$, $p>0$

\begin{mth}zur Lösung
	\begin{enumerate}
		\item Transformiere die DGL mit Hilfe von L und erhalten eine rein algebraische Gl. für $F(p)$ bzgl. $p$.
		\item Löse die algebraischen Gl. nach $F(p)$ auf
		\item Suche die inverse Transformation von $F(p)$, $y(x)=L^{-1}\big(L(y)\big)$
		\item [$\Rightarrow$] Laplace Transformation ist linear
			$L(y'')-2L(y')+L(y)=L(x^3e^x)$\\
		\begin{align*}
			L(y')=\int_{-\infty}^{\infty}dx\frac{dy}{dx}e^{-p x}&=\left[ye^{-px}\right]_0^\infty-\int_{0}^{\infty}dx ye^{-px}(-p)\\
			&=-y(0)+p\int_{0}^{\infty}dx ye^{-px}\\
			&=-y(0)+pL(y)=pL(y)-y(0)
		\end{align*}
	\end{enumerate}
\end{mth}
Analog:
\begin{align*}
	L(y'')&=p^2 L(y)-py(0)-y'(0)\\
	L(x^3e^{x})&=\int_{0}^{\infty}x^2x^2e^{-x(p-1)}dx=\dfrac{6}{(p-1)^4},\quad p-1>0
	\intertext{$\Rightarrow$ Eingesetzt in Gl 1}
	(p^2-2p+1)L(y)&=\frac{6}{(p-1)^4}\\
	L(y)&=\frac{6}{(p-1)^4(p^2-2p+1)}
\end{align*}


\subsection{Erzwungene Schwingungen und Greensche Fkt.}
Schwingungsgleichung
\[M\ddot{\nu}(t)+K\nu(t)=F(t)\]
Normalkoordinaten $Q(t)$ Einführen
\[\ddot Q(t)+kQ=F(t)\]
\[F(t)=O^TM^{-1/2}F(t)\]
\[F(t)=AQ_1(t)+BQ_2(t)+Q_p(t)\]

\paragraph*{Konstruktion von $Q_P(t)$ auf $F(t)$}
Def. Eigenschaften der Greenschen Funktion $G(t,t')$
\[\frac{\partial^2}{\partial t^2}G(t,t')+kG(t,t')=\delta(t-t')\]
Für $Q(t)$ gilt
\[Q_p(t)=\int_{-\infty}^{\infty}dt'G(t,t')F(t')\]

\begin{proof}
	\[\ddot Q_p(t)+kQ_p(t)=\int_{-\infty}^{\infty}dt'\underbrace{\left\{\ddot G(t,t')+kG(t,t')\right\}}_{:=\delta(t-t')}F(t')=F(t)\]
\end{proof}

Es gilt:

\begin{itemize}
	\item $G_{ret}(t,t')=\frac{1}{\sqrt k}\sin(\sqrt k(t-t'))\theta(t-t')$ (retardiert)
	\item $G_{av}(t,t')=\frac{1}{\sqrt k}\sin(\sqrt k(t'-t))\theta(t'-t)$ (avanciert)
\end{itemize}

\begin{proof}
	 Für $G_{ret}(t,t')$ und $k>0$
	$t>t'$: $G_{ret}(t,t')=A\sin(\sqrt{k}(t-t'))+B\cos(\sqrt k(t-t'))$ (Ansatz)\\
	$t<t'$: $G_{ret}(t,t')=0$\\
	Übergangsbedingung für $t=t'$: Sei $\epsilon>0$\\
	Integration von $(2)$ nach $t$
	\[\int_{t'-\epsilon}^{t'+\epsilon}dt\left\{\frac{\partial^2}{\partial t^2}G_{ret}(t,t')+kG_{ret}(t,t')\right\}=\int_{t'-\epsilon}^{t'+\epsilon}dt\delta(t-t')\]
	Im Limes $\epsilon\rightarrow 0$
	\[\frac{\partial}{\partial t}G_{ret}(t'+\epsilon,t')-\frac{\partial}{\partial t}G_{ret}(t'-\epsilon,t')=1\]
	Weper Stetigkeit:
	\[\lim\limits_{\epsilon\rightarrow 0}\frac{\partial}{\partial t}G_{ret}(t'+\epsilon,t')=0\]
	$\Rightarrow $ $B=0$, $A\frac{1}{\sqrt k}$
	\\
	\begin{itemize}
		\item[$\Rightarrow $] Eine Partikuläre Lösung der DGL ist gegeben durch
			\[Q_p(t)=\int_{-\infty}^{t}dt'F(t')\frac{1}{\sqrt k}\sin(\sqrt k(t-t'))\]
			Spezialfall $F(t)=F_0\theta(t')\theta(T-t')\sin\Omega
			t'$ 
		\item [$\Rightarrow$] Eingesetzt in Gl 5
			\[Q_p(t\ge T>0)=\frac{-F_0}{2\sqrt k}\left\{\frac{\sin[(\Omega-\sqrt k)T+\sqrt k t]-\sin(\sqrt k t)}{\Omega-\sqrt k}-\frac{\sin[(\Omega+\sqrt k)T-\sqrt k t]t\sin\sqrt k t}{Omega+\sqrt k}\right\}\]
			mit $\sin\al\sin\beta=\frac{-1}{2}(\cos(\al-\beta)-\cos(\al-\beta)
			)$
			\[\xrightarrow[\Omega\rightarrow \sqrt k]{\null}-\frac{F_0T}{2\sqrt k}\cos\sqrt k t\quad\text{resonante Anregung}\]
	\end{itemize}
\end{proof}

\subsubsection{Konstruktion Greenscher Funktionen für DGL: I) Anfangswertprobleme}

\paragraph*{Beispiel} $\ddot y(t)+\omega^2y(t)=f(t)$, $y(0=\dot y(0)=0)$, $t>0$\\
Sowie $\ddot G(t,t')+\omega^2 G(t,t')=\delta(t-t')$, $G(0,t')=\dot G(0,t')=0$
Durch Multiplikation mit $f(t')$ und integrieren über $t'$:
\[\int_{0}^{\infty}dt'\left(\frac{\partial^2}{\partial t^2}+\omega^2\right)G(t,t')f(t')=\int_{0}^{\infty}dt'\delta(t-t')f(t')\]
Ergibt
\[\left(\frac{\partial^2}{\partial t^2}+\omega^2\right)\int_{0}^{\infty}G(t,t')f(t')=f(t)\]
Transformieren mit Hilfe der Laplace Transformation bzg. p
\[G(p,t')=\frac{e^{-pt'}}{p^2+\omega^2},\quad\text{p>0}\]
\[G(t,t')=\frac{\sin(\omega(t-t'))}{\omega}\theta(t-t')\]

\paragraph*{Fazit}

\begin{enumerate}
	\item Formuliere 6 mit Hilfe von $G(t,t')$ in 7
	\item Löse 7 mciht Laplace Trasnfomration
	\item Trasnformiere G(p,t') zurück zu $G(t,t')$
	\item Interpretiere $y(t)=\int_{-\infty}^{\infty}dt'G(t,t')f(t')$ und erhalte $y(t)$ für ein bestimmtes $f'(t)$.
\end{enumerate}


\subsubsection{Konstruktion Greenscher Funktionen für DGL: II) Randwertprobleme}
\begin{enumerate}
	\item Jede DGL hat die Form $\textfrak{D}y(x)=f(x)$.\\
		Bringe jede DGL 2. Ordnung in die sog. Sturm-Lioville Form
		\[\frac{d}{dx}(p(x)y'(x))+a(x)a(x)=f(x)\]
		Wobei $p(x)>0$ und $a(x)$ stetig, reell
	\item Löse die homogene DGL
	\[	y_n(x)c_1y_1(x)+c_2y_2(x)\]
	\item $y_{ges}=y_n(z)+\int_{a}^{b}dx G(x,z)f(x)$
		\[G(x,z):=c_1 y_1(x)+c_2y_2(x)+\begin{cases}
		\frac{y_1(x)y_2(z)}{p(z)\omega(z)}&\text{für $a<x<z$}\\
		\frac{y_1(z)y_2(x)}{p(z)\omega(z)}&\text{für $a<x<b$}
		\end{cases}\]
		mir $\omega(z)=y_1(z)y'_2(z)-y'_1(z)y_2(z)$ (Wronski Determinante)
\end{enumerate}
\begin{enumerate}
	\item $p(x)=1$
	\item Homogene DGL: $ y''=0$
	\item $G(x,z)=c_1+c_2x+\begin{cases}
		z,&\text{für }0<x<z\\
		x,&\text{für }z<x<1\\
		\end{cases}$
		Bestimme $a_1,c_2$ aus Randbedingungen
		\[\left.\begin{array}{rl}
		G(0,z)&=0\\
		G(1,z)&=0
		\end{array}\right\}\Rightarrow
		\begin{array}{lr}
		c_1&=-z\\
		c_2&=z-1
		\end{array}\Rightarrow G(x,y)=
		\begin{cases}
		x(z-1)&\text{, für }0<x<z\\
		z(x-1)&\text{, für }z<x<1
		\end{cases}\]
		\[\Rightarrow y_P(z)=\]
		%TODO
		\emph{Hier fehlt was!}
\end{enumerate}

\newpage

\section{Hamiltonsche Mechanik}

\subsubsection*{Motivation: Lagrange-Formulierung}

\begin{enumerate}
	\item Grundlegende Variablen $q,q\dot q,t$
	\item Lagrange Funktion $L(q,\dot q,t)$
	\item $\delta\int_{t_0}^{t_1}dt\quad L\big(q(t),\dot q(t),t\big)=0$ (Hamilton Prinzip)
\end{enumerate}
\begin{itemize}
	\item[$\Rightarrow$]$ \frac{d}{dt}\frac{\partial L}{\partial \dot q_j}=\frac{\partial L}{\partial q_j}$ mit $j=1,...,f$\\
	\item[$\Rightarrow$]$ \frac{\partial L}{\partial q_{j_0}}=0$ $\Rightarrow$ $\frac{\partial L}{\partial q_{j_0}}:=p_{j_0}$ ist Erhaltungsgröße
	\item[$\Rightarrow$]Suche nach einer alternativen Formulierung der Lagrange-Menchanik im bezug auf $q,p$ statt $q$ und $\dot q$.
\end{itemize}

\subsection{Legendre Transformation}
Ersetze $f(x)$ durch $g(p)$ mit $P:=\frac{\partial f}{\partial x}$\\
Ziel: FInde eine eindeutige Zuordnung $f(x)\leftrightarrow g(p)$ (Bijektivität)\\

\begin{itemize}
	\item[$\Rightarrow$] Übergang von $L(q,\dot q,t)$ zur Hamilton Funktion $H(P:=\frac{\partial L}{\partial\dot q},q,t)$
\end{itemize}
Gegeben sei $f(x)$ konvex, $f''(x)>0$\\
\[F(x,p):=px-f(x)\]
Es gilt:
\begin{itemize}
	\item $\left.\frac{\partial F}{\partial x}\right|_P=p-\frac{df}{dx}$
	\item $\left.\frac{\partial^2F}{\partial x^2}\right|_P=-\frac{d^2f}{dx^2}<0$, da $f''(x)>0$
\end{itemize}
\begin{itemize}
	\item [$\Rightarrow$] $\max F(x,p)=g(p):=F\big(x(P),P\big)$ mit $P=\frac{\partial f}{\partial x}$
	\item [$\Rightarrow$] d.h. durch die Abbildungsvorschrift $f(x)\rightarrow g(p):=px(p)-f(x(p))$ mit $p:=\frac{\partial f}{\partial x}\rightarrow x(p)$ lässt sich jeder konvexen Funktion $f(x)$ eine Legendre-Trasnformierte $g(p)$ zuordnen.\\
\end{itemize}
Es gilt:
\begin{itemize}
	\item $\frac{dg}{dp}=x(p)+p\frac{dx}{dp}-\frac{df}{dx}\frac{dx}{dp}=x(p)$
	\item $\frac{d^2g}{dp^2}=\frac{dx}{dp}=\left[\frac{d^2 f}{dx^2}\right]^{-1}>0$ $\rightarrow g(p)$ ist auch konvex
	\item Die Legendre Transformation ist involutiv, d.h. die Legendre Transformation von $g(p)$ ist wieder $f(x)$\\
	\underline{Beweis:} $\max G(x,p)=p(x)x-g(p(x)):=p(x)x-\{p(x)x-f(x)\}=f(x)$
	\item $-g(P(x))$ ist die Ordinate $\Psi$ des Punktes $x=0=x_0$ der Tangente an die Kurve $f(x)$ an der Stelle $x_0$
		\[\left.\frac{df}{dx}\right|_{x_0}=\frac{f(x_0)-\Psi}{x_0-0}\Leftrightarrow -\Psi=x_0\left.\frac{df}{dx}\right|_{x_0}-f(x_0)\equiv g(p(x_0))\]
	\item Für $f(x)$ von mehreren Unbekannten $x_j$, $j=1,...,d$ gilt:
		\[g(p)=\sum_{j=1}^{d}p_jx_j(p)-f(x(p))\text{ mit }p_j:=\frac{\partial f}{\partial x^j}\rightarrow x_j(p_j)\]
		$\frac{\partial^2 f}{\partial x_j\partial x_i}>0$ (positiv definit)\\
		Analog gilt für die Umkehrung:\begin{align*}
		f(x)&=\sum_{j=1}^{d}p_j(x)x_j-g\big(p(x)\big)\\
		x_j&=\frac{\partial q}{\partial p_j}p\rightarrow p_j(x_j)\\
		\frac{\partial ^2 q}{\partial p_j\partial p_j}&=\left[\frac{\partial ^2 f}{\partial x_j\partial x_j}\right]^{-1}>0
		\end{align*}
	\item Youngsche Ungleichung
		\[F(x,p):=\sum_{j=1}^{d}p_jx_j-f(x)\leq g(p)\equiv F\big(x(p),p\big)\equiv\max_x F(x,p),\text{ für }\frac{\partial^2 f}{\partial x^2}\]
		\begin{figure*}[t]
			\begin{center}
			\begin{tikzpicture}
				\draw[->](0,0)--(2,0)node[below]{$x(t)$};
				\draw[->](0,0)--(0,2)node[left]{$p(t)$};
				\draw[variable=\x,domain=0:1,samples=500] plot ({\x * cos( 20*\x r)+1},{\x * sin(20*\x r)+1});
			\end{tikzpicture}
			\end{center}
			\caption{Phasenraum-Darstellung}
			\label{fig:phase}
		\end{figure*}
	\item $dg=\sum_{i=1}^{d}\left\{dx_ip_i+x_idp_i-\frac{\partial f}{\partial x_i}dx_i\right\}=\sum_{i=1}^{d}p_idp_i\Leftrightarrow \frac{\partial g}{\partial p_i}=x_i$\\
		$df=\sum_{i=1}^{d}\left\{dx_ip_i+x_idp_i-\frac{\partial g}{\partial p_i}dp_i\right\}=\sum_{i=1}^{d}p_idp_i\Leftrightarrow \frac{\partial F}{\partial x_i}=p_i$
	\item Die Legendre-Funtion (engl. Hemiltonian) mit $p_i=\frac{\partial L}{\partial\dot q_j}(q,\dot q,t)\rightarrow \dot q(q,p,t)$ für $\frac{\partial^2L}{\partial \dot q_i\dot q_j}>0$
	\item[$\Rightarrow$] $\frac{\partial L}{\partial q_{j_0}}=0$ $\rightarrow $ $\frac{\partial H}{\partial q_{j_0}}=0$ (zyklische Koordinaten)
\end{itemize}


% TODO Beispiel hier?
%VL 8.12.2015
\subsection{Die Hamiltonschen Bewegungsgleichungen}
Es gilt:
\[\frac{d}{dt}\left(\frac{\partial L}{\partial\dot q_i}\right)=\frac{\partial L}{\partial q_i}\quad\text{mit Anfangsbedingungen}\]
Es gilt:
\[\begin{array}{rl}
\dot q_j(t)&=\frac{\partial H}{\partial p_j}(p,q,t)\\
\dot p_j(t)&=-\frac{\partial H}{\partial q_j}(p,q,t)
\end{array}\rightarrow \text{$(p(t),q(t))$ für gegebenen Anfangsbedingungen $(p(t_0),q(t_0))$}\]
Hamiltonsche Bewegungsgleichungen

\begin{itemize}
	\item $\dif{H}{q_j}=-\dif{L}{q_j} a$
	\item $\dif{H}{t}=-\dif{L}{t}$
	\item $\frac{dH}{dt}\big(p(t),q(t),t\big)=\dif{H}{t}\big(p(t),q(t),t\big)$\\
		Lösung der Hamilton-Bewegungsgleichung\\
		$\Rightarrow$ für $\dif{H}{t}(p,q,t)=0\Rightarrow H\big(p(t),q(t),t\big)$ ist Erhaltungsgröße (Energie)
\end{itemize}

\begin{proof}
	Freiheitsgrade $f=:H(p,q,t)=\sum_{j=1}^{f}\left[p_j\dot q_j(q,p,t)\right]-L\big(q,\dot q(q,p,t),t\big)$
	
	\begin{align*}
		dH(p,q,t)|_p&=\sum_{j=1}^{f}\left.\left(\dif{H}{p_j}dp_j+\dif{H}{q_j}dq_j+\dif{H}{t}dt\right)\right|_P\\
		\frac{df}{dx}(x)\equiv f'(x)&\equiv\sum_{j=1}^{f}\left.\left( dp_j\dot q_j+p_jd\do q_j-\dif{L}{q_j}-\dif{L}{\dot q_j}d\dot q_j-\dif{L}{t}dt\right)\right|_T
		\intertext{mit $\left(p_j-\dif{L}{\dot q_j}\right)d\dot q_j=0$}
		\Leftrightarrow \dot q_j=\dif{H}{p_j}
		\intertext{Es folgt daraus dass}
		\dif{H}{q_j}=-\dif{L}{q_j}\quad\text{Hamilton I}\\
		\dif{H}{t}=-\dif{L}{t}\quad\text{Hamilton II}
		\intertext{Weiterhin}
		\frac{dH}{dt}&=\sum_{j=1}^{f}\left(\dif{H}{p_j}\underset{\dot p_j}{=-\dif{H}{q_j}}+\dif{H}{q_j}\underset{\dot q_j}{=\dif{H}{p_j}}\right)+\dif{H}{t}=\dif{H}{t}
	\end{align*}
	
\end{proof}

Es gilt: Sei $q_1$ Zyklische Koordinate\\
$\dif{L}{q_1}=0=-\dif{H}{q_1}$ $\Rightarrow$ $p_1$ Erhaltungsgröße: $H(p_1=c,p_2,...,p_f)$\\
%TODO

\subsection{Das modifizierte Hamiltonsche Prinzip}
Wenn $p=\dif{L}{\dot q_j}(q,\dot q,t)$ (lokal) bijektiv ist

\begin{align*}
	S:&=\int_{t_0}^{t_1}\left\{\sum_{j=1}^{f}p_jdq_i-H(q,p,t)dt\right\}\\
	&=\int_{t_0}^{t_1}dt\left\{\sum_{j=1}^{f}p_j(t)\dot q_j(t)-H(p(t),q(t),t)\right\}\le\int_{t_0}^{t_1}dtL\big(q(t),\dot q(t),t\big)\quad\text{Youngsche Ungleichung}
\end{align*}

%VL 10.12.2015
\begin{align*}
	\delta S[\gamma]=\delta \int_{t_0}^{t_1}dt\left\{\sum_{j=1}^{f}p_j(t)\dot q_j(t)-H(p(t),q(t),t)\right\}=0
\end{align*}
Wobei der Weg $\gamma:t\mapsto(p(t),q(t))$. \\
Es gilt: Die Variation ist gleich 0:
\[\delta q(t_0)=\delta q(t_1)=0\]
bzw. $p(t_0),p(t_1)$ beliebig
\begin{proof}
	betrachte virtuelle Verrückung $\big(\delta p(t_1),\delta q(t_1)\big)$
	\begin{align*}
		\delta S[\gamma]&=\int_{t_0}^{t_1}dt\left\{\sum_{j=1}^{f}\left(\delta p_j(t)\dot q_j(t)+\underbrace{p_j(t)\delta\dot q_j(t)}_{=\frac{d}{dt}\left(p_j(t)\delta q_j(t)-\delta q_j(t)p_j(t)q_j(t)\right)}-\dif{H}{p_j}\delta p_j(t)-\dif{H}{q_j}\delta q_j(t)\right)\right\}\\
		&=\left.\sum_{j=1}^{f}p_j\delta q_j(t)\right|_{t_0}^{t_1}+\int_{t_0}^{t_1}dt\left\{\sum_{j=1}^{f}\delta q_j(t)\left(\dot q_j(t)-\dif{H}{p_j}\right)+\delta q_j(t)\left(-\dot p_j(t)-\dif{H}{q_j}\right)\right\}\underset{\text{da $\delta q(t_0)=\delta q(t_1)=0$}}{\overset{!}{=}}0\\
		&\Leftrightarrow\\
		&\dot q_j=\dif{H}{q_j},\quad \dot p_j=-\dif{H}{q_j}\qedhere
	\end{align*}
\end{proof}


\subsection{Phasenraum, Zustände, physikalische Variable}

\paragraph*{Ausgangspunkt} $H(p,q,t)$ kodiert mechanisches System\\
Raum $=\{(p,q,t)\}$ Phasenraum\\
$\widetilde{\text{Raum}}:=\{(p,q,t)\}$ erweiterter Phasenraum

\[\begin{array}{c}
\dot q_j=\dif{H}{p_j}\\
\dot p_j=\dif{H}{p_j}
\end{array}\Leftrightarrow\dot z^k=\sum_{l=1}^{2f}S_{kl}\dif{H}{z_l}=V_k(z)\]
\[V_k=\sum_{j=1}^{2f}\overset{\substack{\text{konvar.}\\\text{Tenstor}\\\text{2. Stufe}\\\text{ }}}{S_{kl}}\underset{\substack{\text{konvariantes}\\\text{Vektorfeld}}}{\dif{H}{z_l}}\text{kontravariantes Vektorfeld}\]

\subparagraph*{Trajektorien} $\dot z_k(t)=\sum_{l=1}^{2f}S_{kl}\dif{H}{z_l}(z)$ Hamiltonsche Bewegungsgleichungen
\subparagraph*{Zustand}(rein) $z=(p,q)$ charakterisier das mechanische System Vollständig
\subparagraph*{Physikalische Variable} Funjtion über dem Phasenraum (z.B. $H(z(t),t)$)
\subparagraph*{Physikalische Variable} (Zustandsgröße, physikalische Variable) $H(z_0,t_0)$
\subparagraph*{gemischte Zustande}Charakterisierung durch Wahrscheinlichkeitsdichteverteilung

\[P(z,t)\ge 0\quad \int dz P(z,t_0)=1\]
\[\langle F\rangle_{P(t_0)}=\int dz P(z,t_0)F(z,t_0)\]

\subsubsection*{Zeitliche Entwicklung und Poisson Klammern:}
Sie $F(p,q,t)$ eine Physikalische Variable\\
\[\dot q_l=\{q_l,H\}=\dif{H}{p_l}\quad\dot p_l=\{p_l,H\}=\dif{H}{q_l}\]

\begin{align*}
	\frac{dF}{dt}\big(p(t),q(t),t\big)&\underset{\text{H-B-Gl}}{=}\sum_{l=1}^{f}\left\{\dif{F}{p_l}\dot p_l+\dif{F}{q_l}\dot q_l\right\}+\dif{F}{t}\big(p(t),q(t),t\big)\\
	&=\quad\dif{F}{t}\big(p(t),q(t),t\big)+\{F,H\}
	\intertext{wobei $\{F,G\}=\sum_{l=1}^{f}\left(\dif{F}{q_l}\dif{G}{p_l}-\dif{F}{p_l}\dif{G}{q_l}\right)$} 
	&=\sum_{k=1}^{f}\left(\dif{F}{q_l}\dif{H}{p_l}-\dif{F}{p_l}\dif{H}{q_l}\right)
\end{align*}

%VL 15.12.2015
Neue Struktur: Über die Menge der physikalischen Variablen $\{F(z)\}$ ist eine algebraische Struktur definiert
\begin{itemize}
	\item gewöhnliche Multiplikation $(FG)(z):=F(z)G(z)$ ist kommutativ
	\item Poisson-Klammern $\{F,G\}:=R(z)(:=F\circ G)$ ist nicht kommutativ
\end{itemize}

\paragraph*{Eigenschaften der Poisson-Klammern}
\begin{itemize}
	\item $\{F,c\}=0$ für $\dif{c}{p_i}=\dif{c}{q_i}=0$ für alle $i=1,...,f$
	\item $\{F,\al G_1+\beta G_2 \}=\al\{F,G_1\}+\beta\{F,G_2\}$ für $\al,\beta\in\R$
	\item $\{F,G\}=-\{G,F\}\Rightarrow \{F,F\}=0$ antikommutativ
	\item $\{F,G_1G_2\}=\{F,G_1\}G_2+G_1\{F,G_2\}$ (Produktregel) 
	\item $\{F,\{G_1,G_2\}\}+\{G_1,\{G_2,F\}\}+\{G_2,\{F,G_1\}\}$ 
\end{itemize}
$\rightarrow $\underline{Quantenmechanik}: Heisenberg: $p,q$ sind Matrizen $\rightarrow$ lineare Vektorraum Transformationen.
\[\{F,G\}\rightarrow [\hat F,\hat G]:=\hat F\hat G-\hat G\hat F\]

\paragraph*{Beispiel:} Poisson-Klammer-Strukturen $a,b\in\R^3$
\begin{align*}
	\vec a\times\vec b&:=\{\vec a,\vec b\}&\vec a\times(\vec b\times\vec c)+\vec b\times(\vec c\times\vec a)+\vec c\times(\vec a\times \vec b)&=0
\end{align*}

\paragraph*{Entwickelung}
\begin{align*}
	F(p(t),q(t))&=F\big(p(t),q(t)\big)+(t-t_0)\underbrace{\left.\frac{dF}{dt}\right|t_0}_{=-\{H,F\}_{t_0}}+\frac{1}{2}(t-t_0)^2\underbrace{\left.\frac{d^2F}{dt^2}\right|_{t_0}}_{(-1)^2\{H,\{H,F\}_{t_0}}+...\\
	&=\sum_{n=0}^{\infty}\frac{(-1)^n(t-t_0)^n}{n!}\{\underbrace{H\{H\{...\{H,F}_{\text{n-mal}}\}...\}|_{t_0}
\end{align*}


\subsubsection{Erhaltungsgrößen}
\begin{align*}
	=\overset{!}{=}\frac{dF}{dt}\big(p(t),q(t),t\big)=\{F,H\}+\dif{F}{t}
\end{align*}
Es gilt der  Poissonsche Satz: Seien $F_1$ und $F_2$ Erhaltungsgrößen: $\Rightarrow$ $\{F,G\}$ist Erhaltungsgröße
\begin{proof}
	\begin{align*}
		\frac{d}{dt}\{F_1,F_2\}&=\{\{F_1,F_2\},H\}&+&&\dif{\null}{t}\{F_1,F_2\}\\
		&=-\{H,\{F_1,F_2\}\}&+&&\{\dif{F_1}{t},F_2\}+\{F_1,\dif{F_2}{t}\}\\
		\text{Mit Jacobi-Id} &=\{F_1,\{F_2,H\}\}+\{F_2,\{H,F_1\}\}&+&&\{\dif{F_1}{t},F_2\}+\{F_1,\dif{F_2}{t}\}\\
		&=\{F_1,\underbrace{\{F_2,H\}+\dif{F_2}{t}}_{=0}\}&+&&\{F_2,\underbrace{\underbrace{\{H,F_1\}}_{=-\{F_1,H\}}-\dif{F_1}{t}}_{=0}\}\\
		&=0
	\end{align*}
\end{proof}


\subsubsection*{Erhaltungsgröße eines dreien nicht relativistischen Massepunkts}
Zusammenhänge:
\begin{align*}
	\{q_i,p_k\}&=\delta_{ik}\\
	\{q_i,q_k\}&=\{p_i,p_k\}=0
\end{align*}

\[H=\sum_{i=1}^{3}\frac{p_ip_i}{2m}\]
\begin{enumerate}
	\item Impulserhaltung
		\[\dot p_i=\{p_i,H\}=\frac{1}{2m}\sum_{k=1}^{3}\{p_i,p_k,p_k\}=\frac{1}{2m}\sum_{k=1}^{3}\underbrace{\{p_i,p_k\}}_{=0}p_k+p_k\underbrace{\{p_i,p_k\}}_{=0}=0\]
	\item Energieerhaltung
		\[\frac{dH}{dt}=\{H,H\}=0\quad\text{da $\dif{H}{t}=0$}\]
	\item Drehimpulserhaltung
		\begin{align*}
		\vec L&=\vec x\times \vec p&L_i=\sum_{l,k=1}^{3}\epsilon_{ilk}x_lp_k
		\end{align*}
		\begin{align*}
		\frac{dL_i}{dt}=\{L_i,H\}=\frac{1}{2m}\sum_{k=1}^{3}\{L_ip_kp_k\}=\frac{1}{m}\sum_{k=1}^{3}\{L_i,p_k\}p_k=\frac{1}{m}\sum_{k,b=1}^{3}\epsilon_{ikb}p_kp_b=0
		\end{align*}
\end{enumerate}
7 Erhaltungsgrößen in 6 dim $\Rightarrow$ min 2 sind Abhängig
%VL 17.12.2015
%TODO
\emph{Hier fehlt was!}


\subsection{Die Poincare-Cartan Invariante}
erweiterter Phasenraum $\{(p,q,t)\}$,geschlossener Weg $C:\la\in[0,1]\mapsto\big(p(\la),q(\la),t(\la)\big)$.
\begin{align*}
	I(C)&=\oint_C\left\{\sum_{i=1}^{f}p_1(\la)\frac{dq_i}{d\la}-H\big(p(\la),q(\la),t(\la)\big)\frac{dt}{d\la}\right\}d\la\\
	&=\oint_C\sum_{i=1}^{f}p_idq_i-Hdt\\
	&=I(C')
\end{align*}
Ist äquivalent zu
\begin{align*}
	\dot p_i&=\dif{H}{p_i}&\dot q_i=-\dif{H}{p_i}
\end{align*}

\paragraph*{Konsequenz}
Die Koordinaten $(P,Q,T),K(P,Q,T),S(P,Q,T)$ hängen von $(p,q,t)$ und $H(p,q,t)$ ab, sodass wenn
\[\sum_{i=1}^{f}p_idq_i-Hdt=\sum_{i=1}^{f}P_idQ_i-KdT+dS\]
Dann folgt durch die Poincare-Cartan Invariante (P-C-I) und durch $\oint_CdS=0$ die Äquivalenz zwischen
	\[\begin{array}{rcl}
	\frac{dq_i}{dt}&=\dot q_i&=\dif{H}{p_i}\\
	\frac{dp_i}{dt}&=\dot p_i&=\dif{H}{q_i}
	\end{array}\Leftrightarrow\begin{array}{rl}
	\frac{dQ_i}{dT}&=\dif{K}{P_i}\\
	\frac{dP_i}{dT}&=-\dif{K}{Q_i}
	\end{array}\]

\paragraph*{Beweisidee}betrachte eine beliebige (glatte) Abbildung $s\in[0,...)$ und $\big(P(p,q,t),Q(p,q,t),T(p,q,t)\big)$ und $\tilde H(P,Q,T):=H\big(p(P,Q,T),q(P,Q,T),t(P,Q,T)\big)$. Wir betrachten die Reihenentwicklungen

\begin{align*}
	P_i&=p_i+s\dot P+O(s^2)\\
	Q_i&=q_i+s\dot Q+O(s^2)\\
	T&=t+s\dot T+O(s^2)
\end{align*}
Durch einsetzen diese in die PCI
\begin{align*}
	I(C')&=\oint_{C'}\sum_{i=1}^{f}P_i~dQ_i-\tilde H~dT\\
	&=\oint\sum_{i=1}^{f}(p_q+s\dot P_i)(dq_i+sd\dot Q_i)-\left(H+\sum_{i=1}^{f}\left(\dif{H}{p_i}s\dot P_i+\dif{H}{q}s\dot Q_i\right)+\dif{H}{t}s\dot T\right)\left(dt+sd\dot T\right)+O(s^2)\\
	&=\left(\oint_{C'}\sum_{i=1}^{f}p_i~dq_i-H~dt\right)\\&\quad+s\oint_{C'}\left\{\sum_{i=1}^{f}\left(\dot P_i~dq_o+p_i~d\dot Q_i-\dot Q_i~dp_i-\dif{H}{p_i}\dot P_i~dt-\dif{H}{p_i}\dot Q_i~dt-\dif{H}{t}\dot Tdt\right)-Hd\dot T\right\}+O(s^2)
	\intertext{wobei $p_i~d\dot Q_i=d(p_i\dot Q_i)-\dot Q_i~dp_i$ und $Hd\dot T=d(H\dot T)-\dot TdH$}
	\intertext{Durch Ordnen der Summe}
	&=\oint_{C'}\left(\sum_{i=1}^{f}p_i~dq_i-H~dt\right)+\underbrace{\oint_{C'}d(p_iQ_i)-d(\tilde H\dot T)}_{=0}\\
	&\quad+s\underbrace{\oint\sum_{i=1}^{f}\left\{\dot P_i\left(dq_i-\dif{H}{p_i}dt\right)+\dot Q_i\left(-dp_i-\dif{H}{q_i}dt\right)+\dot T\left(-\dif{H}{t}+d\tilde H\right)\right\}}_{=\tilde{I}}+O(s^2)\\
\end{align*}

%VL 12.01.2016
Fall 1: Gelte für die Kurve, dass $dt\equiv\frac{dt}{d\la}d\la=0$
\begin{align*}
	\tilde I=\oint_C\sum_{i=1}^{f}(\dot P_i+\dot T\dif{dH}{dq_i})dq_i+(-\dot Q_i+\dot T\dif{H}{p_i})dp_i\overset{!}{=}0
\end{align*}
Ist äquivalent zu
\begin{align*}
	\frac{dp_i}{dt}\equiv\frac{\left.\frac{dP}{ds}\right|_{s=0}}{\left.\frac{dT}{ds}\right|_{s=0}}=\frac{\dot P_i}{\dot T}=-\dif{H}{q_i}&&\frac{\dot Q_i}{\dot T}=\dif{H}{p_i}=\frac{dq_i}{dt}
\end{align*}

\subsection{Kanonische Transformationen} %4.6
Sei $z$ die ... 
\begin{align*}
	z_i=(p,q)\\
	\dot z_i&=\sum_{k=1}^{2f}S_{ik}\dif{H}{z_k}=\{z_i,H\}
\end{align*}
und gebe es eine Transformation $Z(z)$, sodass
\[Z(z)\mapsto \dot{Z}_I\tilde{S}"{ik}\dif{\tilde{H}}{Z_k}=\{Z_i,\tilde H\}\]
\begin{align*}
	\tilde S_{ik}=\sum_{a,b=1}^{2f}\dif{Z_i}{z_a}A_{ab}\dif{Z_k}{z_b}
	\tilde{H}(Z:=H(z(Z)))
\end{align*}
Wenn $S_{ik}=\tilde S_{ik}$, dann herrscht formale ``Gleichheit'' der Hamiltonschen Bewegungsgleichungen.\\
\\
Kanonische Transformationen (erhalten die Poissonklammern) und es gilt $S_{ik}=\{z_i,z_k\}$
\paragraph*{Eigenschaften kanonischer Transformationen:} $z\leftrightarrow Z$ ist kanonisch $\Leftrightarrow$
\begin{enumerate}
	\item $\int_{G'}\sum_{i,k=1}^{2f}S_{ik}^{-1}~dZ_idZ_k=\int_{G}\sum_{i,k=1}^{2f}S^{-1}_{ik}dz_idz_k$
		für alle $G(G')$ im Phasenraum
		\begin{proof}[Beweisidee]
			Sei $z\leftrightarrow Z$ beliebig:
			\begin{align*}
			\int_{_G}\sum_{i,k=1}^{2f}S_{ik}~dz_i~dz_k&=\int_{G'}\sum_{i,k=1}^{2f}\tilde{S}_{ik}~dZ_i~dZ_k\\
			&=\int_{G'}\sum_{i,k=1}^{2f}S_{ik}~dZ_i~dZ_k
			\end{align*}
		\end{proof}
	\item $\oint_{\partial G}\sum_{i=1}^{f}p_i~dq_i=\oint_{\partial G'}\sum_{i=1}^{f}P_i~dQ_i$ für alle $\partial G(\partial G')$
		\begin{proof}[Beweisidee]
			\begin{align*}
				\int_G\sum_{i,k=1}^{2f}S_{ik}\dif{z_i}{u}\dif{z_k}{v}~du~dv&=\sum_{i=1}^{f}\int ~du~dv\left(\dif{p_i}{u}\dif{q_i}{v}-\dif{p_i}{v}\dif{q_i}{u}\right)\\
				\intertext{Wir wählen $u_<\le u\le u_>$ und $v_<\le v\le v_>$}
				&=\sum_{i=1}^{f}\left\{\int dv~p_i\left.\dif{q_i}{v}\right|_{u_<}^{u_>}
				-\int du~dv~p_i\dif{^2q_i}{u\partial v}\right.\\
				&\left.\quad-\int du~dv~p_i\left.\dif{^2q_i}{u\partial v}\right|_{v_<}^{v_>}
				+\int du~dv~p_i\dif{^2q_i}{u\partial v}\right\}\\
				&=\sum_{i=1}^{f}\oint p_i~dq_i
			\end{align*}
	\end{proof}
	\item Aus (2) folgt:\\
		Wenn $\sum_{i=1}^{f}p_i~dq_i=\sum_{i=1}^{f}P_i~dQ_i+d\tilde{S}$ für die erzeugende Funktion einer kanonischen Transformation $\tilde{S}(P,Q)=S\big(p(P,Q),q(P,Q)\big)$, \\
		dann gilt $z\leftrightarrow Z$ ist kanonisch.
		\begin{proof} Es existiert eine erzeugende Funktion kanonischer Transformationen $S(P,Q,t)=S\big(P(p,q,t),Q(p,q,t),t\big)$ sodass:
			\begin{align*}
				\sum_{i=1}^{f}p_i~dq_i&=\sum_{i=1}^{f}\left(P_i~dQ_i+\dif{S}{P_i}dP_i+\dif{S}{Q_i}dQ_i\right)\\
				\oint_{\partial G}\sum_{i=1}^{f}p_i~dq_i
				&\equiv\oint_{\partial G}\sum_{i=1}^{f}P_i~dQ_i+
				\underbrace{\oint_{\partial G'}\sum_{i=1}^{f}\left(\dif{S}{P_i}dP_i+\dif{S}{Q_i}dQ_i\right)}_{=0}
			\end{align*}
			Daraus folgt, dass
			\begin{align*}
				I(c)&:=\oint_C\sum_{i=1}^{f}p_i~dq_i-H(p,q,t)dt\\
				&=\oint_{C'}\sum_{i=1}^{f}P_i~dQ_i-(\underbrace{\overbrace{H\big(p(P,Q,t),q(P,Q,t),t\big)}^{:=\tilde{H}(P,Q,t)}+\dif{S}{t}}_{=K(P,Q,t)})dt\\
				&\quad+\underbrace{\oint_C\sum_{i=1}^{f}\left(\dif{S}{P_i}dP_i+\dif{S}{Q_i}dQ_i\right)+\dif{S}{t}dt}_{=0}
			\end{align*}
			%TODO (in VL 14.01.2015)
		\end{proof}
\end{enumerate}

\begin{bem}
	Es gilt:\\
	Hamilton-Dynamik mit $\big(p,q;H(p,q,t)\big)$ $\Leftrightarrow$ Hamilton-Dynamik mit $\big(P,Q;k(P,Q,t)\big)$\\
	$(p,q)\leftrightarrow(P,Q)$ mit $K(P,Q,t)=H\big(p(P,Q,t),q(P,Q,t)+\dif{\tilde S}{t}(P,Q,q)\big)$ ist kanonisch (Newtonsche Zeilt $t$ wir nicht Tranformiert)
\end{bem}


\subsubsection*{Verbindung zur Poincare-Cartan Invariante}
Sei eine kanonische Transformation $z\leftrightarrow Z$ bzw $(p,q,t)\leftrightarrow(P,Q,T)$ gegeben, dann gilt
\[\oint_C\sum_{i=1}^{f}p_i~dq_i-H~dt=\oint_{C'}\sum_{i=1}^{f}P_i~dQ_i-\tilde{H}~dt+d\tilde{S}\]
und
\[\sum_{i=1}^{f}p_i~dq_i=\sum_{i=1}^{f}\left(P_i~dQ_i+\dif{\tilde S}{P_i}dP_i+\dif{\tilde S}{Q_i}dQ_i\right)\]
sodass \fbox{$H=\tilde{H}+\dif{\tilde S}{t}$}


%VL 14.01.2016
\subsubsection{Erzeugende Funktionen und kanonische Transformationen} %4.6.1
Ausgangspunkt: kanonische Transformation $(p,Q)\leftrightarrow(P,Q)$ mit erzeugender Funktion $\tilde S(P,Q,t)\equiv S(p,q,t)$.\\
Lokal ist $\det\dif{(P,Q)}{(p,q)}\neq 0$. Existiert eine Transformationsformel?\\
\\
Sei $(p,q)\leftrightarrow (P,Q)\leftrightarrow(q,Q)$. Daraus definieren wir $S_2(q,Q,t):=\tilde S(P(q,Q,t),Q,t)$.
\[\sum_{i=1}^{f}\left\{p_i~dq_i-P_i~dQ_i-\dif{S_2}{q_i}dq_i-\dif{S_2}{Q_i}dQ_i\right\}=0\]
$(q,Q)$ sind Koordinaten. Daraus folgt $dq_i,dQ_i$ sind linaer unabhängig:
\[p_i=\dif{S_2}{q_i}(q,Q,t)\quad\quad P_i=-\dif{S_2}{Q_i}(q,Q,t)\]
Wahl $(p,q)\leftrightarrow(P,Q)$ ist nicht immer möglich (zB. id: $q_i=Q_i$, $p_i=P_i$)
%TODO


%VL 19.01.2016
\subsubsection{Kontinuierliche kanonsche Transformationen} %4.6.2
Sei $g_\epsilon(q,p)\rightarrow (P,Q)$ stetig aus den Einheitstransformationen erzeugbar, also erzeugt aus
\[S_3(P,q,t,\epsilon)=\sum_{i=1}^{f}P_iq_i+\epsilon F(P,q,t,\epsilon)\]
Für die erzeugende Funktion $F(p,q,t)$ gilt
\begin{align*}
	\frac{dp_i}{d\epsilon}&=-\dif{F}{q_i}&\frac{dq_i}{d\epsilon}&=\dif{F}{p_i}
\end{align*}
\begin{align*}
	g_\epsilon:&&p_i&=\dif{S_3}{q_i}=P_i+\epsilon\dif{F}{q_i}(P,q,t,\epsilon)\\
	&&\Rightarrow &\frac{dp_i}{d\epsilon}=\left.\frac{P_i-p_i}{\epsilon}\right|_{\epsilon=0}=-\dif{F}{q_i}(P,q,t,\epsilon=0)=-\dif{F}{q_i}(p,q,t,\epsilon=0)\\
	&&Q_i&=\dif{S_3}{P_i}=q_i+\epsilon\dif{F}{qPi}(P,q,t,\epsilon)\\
	&&\Rightarrow &\frac{dq_i}{d\epsilon}=\left.\frac{Q_i-q_i}{\epsilon}\right|_{\epsilon=0}=-\dif{F}{P_i}(P,q,t,\epsilon=0)=-\dif{F}{p_i}(p,q,t,\epsilon=0)
\end{align*}

\begin{bem}
	Einene Speziallfall der Erzeugenden Funktion stellt die Zeitliche Entwickelung dar:\\
	Wenn $\epsilon\rightarrow t$ dann ist $F\rightarrow H$ einen kontinuierliche Kanonische Transformation.
\end{bem}


\subsubsection{Symmetrietransformationen}
\paragraph*{Definition einer Symmetrietransformation}
\[\xymatrix{
	z=(p,q)\ar[r]^{h_t}_{H}\ar[d]_{g_\epsilon}^{F}& (p_t,q_t)\ar[d]_{g_\epsilon}^{F}\\
	z_\epsilon=\big(P(\epsilon),Q(\epsilon)\big)\ar[r]^{H}_{h_t}&\big(P_t(\epsilon),Q_t(\epsilon)\big)
}
\Leftrightarrow\{F,H\}=0\]
Sei $F(p,q)$ explizit Zeitunabhängig $\Rightarrow\frac{dF}{dt}=\{F,H\}=0$
\begin{proof}
	\begin{align*}
		h_t\big(g_\epsilon(z)\big)=M_1\\
		h_t\big(h_t(z)\big)=M_2\\
	\end{align*}
	Durch verwenden der Jacobi Identität:
	\[M_1-M_2=\epsilon t\{z,\{F,H\}\}+O(\epsilon^2,t^2,\epsilon t)\]
\end{proof}


\paragraph{Beispiele:} freie Teilchen mit $H(p,q,t))\frac{\vec p^22}{2m}$\\
Erhaltungsgröße $\overset{?}{\Rightarrow}$ Symmetrietransformation\\
\[\{H,p_i\}=0\quad\Rightarrow\quad F_{i_0}(p,q,t)\equiv p_{i_0}\]
\begin{align*}
	\frac{d\vec p}{d\epsilon}&=\vec e_{i_0}\times\vec p\\
	\frac{d\vec x}{d\epsilon}&=\vec e_{i_0}\times\vec{x}\\
\end{align*}
\begin{align*}
	\frac{dq_i}{d\epsilon}&=\dif{F_{i_0}}{p_i}=\delta_{ii_0}\\
	\frac{dp_i}{d\epsilon}&=-\dif{F_{i_0}}{q_i}=0\\
\end{align*}
%TODO


%Nachtrag: VL 11.020.2016
\subsubsection{Hamilton-Jakobische Differentialgleichung}
Es gilt: nicht partielle DGL 1. Ordnung $\leftrightarrow$ Hamiltonisches System von gewöhnlichen DGLs\\
kanonische Transformation in ein Hamilton System: $(p,q)\leftrightarrow(P,Q)$.\\
Es existier eine erzeugende Funktion $S(P,Q,p,q,t)$, sodass
\begin{align*}
		p&=\dif{S}{q}&Q&=\dif{S}{P}&P&=-\dif{S}{Q}
	\end{align*}
	\begin{align*}
		\begin{array}{rl}
		\dot{q}&=\dif{H}{p}\\
		\dot q&=-\dif{H}{q}
		\end{array}
		\overset{S}{\leftrightarrow}
		\begin{array}{rl}
		\dot{Q}&=\dif{K}{P}\\
		\dot P&=-\dif{K}{P}
		\end{array}
\end{align*}
Es folgt die H-J-DGl:
\[K=0=H(\dif{S}{q},q,t)+\dif{S}{t}\]
\begin{exm}
	Der Freie Massenpunkt:
	\[H(\vec p,\vec x,t)=\frac{\vec p^2}{2M}\]
	H-J-DGl:
	\[H(\dif{S}{x_i},x_j,t)+\dif{S}{t}=K=\frac{1}{2}M(\vec{\nabla}_{\vec x}S)^2+\dif{S}{t}\overset{!}{=}0\]
	Damit das problme eindeutig gelöst werden kann nehemn wir an, dass $S(\vec P,\vec x,t=0)$ vorgegeben ist.\\
	Nun suchen wir eine Lösung mit dem Separationsansatz: $S(\vec P,\vec x,t)=W(\vec P,\vec x)+\eta t$.\\
	Es ergibt sich eine reduziertes Problem:
	\[\frac{1}{2M}(\vec \nabla_{\vec x}S)^2+\eta =0\]
	Es folgt, als mögliche Lösung $W(\vec P,\vec x)=\vec P\vec x$, sodass $\eta=-\frac{\vec P^2}{2M}$
	\[S(\vec P,\vec x,t)=\vec P\vec x-\frac{\vec P^2}{2M}t\]
	erfüllt die H-J-DGl mit $S=\vec P\vec x$, sodass
	\begin{align*}
		p_i&=\dif{S}{x_i}=P_i\\
		Q_i&=\dif{S}{P_i}=x_i-\frac{P_i}{M}t\\
		x_i&=Q_i+\frac{P_i}{M}\\
		p_i&=P_i
	\end{align*}
	%TODO
\end{exm}


\paragraph*{Hamilton gl zu H-J-DGl}
Löse die Hamiltonsche Bewegunggelichung mit vorgegebenenm $S_0(\vec x)$
\subsubsection{Auf den Spuren von Erwin Schrödinger}
Schrödinger Gleichungen: Punktquant im äußeren Potenzial
\[\left\{-\frac{\hbar^2}{2M}\Delta_{\vec x}+V(\vec x,t)\right\}\Psi(\vec x,t)=i\hbar\dif{\Psi}{t}(\vec x,t)\]
Kurzwellenasymptotik $\hbar\rightarrow 0$ für kleine Wellenlängen
\begin{align*}
	\Psi(\vec x,t)&=Ae^{\frac{i}{\hbar}S(\vec x,t)}\\
	(\vec \nabla\Psi)(\vec x,t)&=\frac{i}{\hbar}(\vec{\nabla}S)\Psi(\vec x,t)\\
	(\Delta\Psi)(\vec x,t)&=\frac{i}{\hbar}(\Delta S)\Psi(\vec x,t)+(\frac{i}{\hbar}\vec{\nabla}S)^2\Psi(\vec x,t)\\
	|(\vec\nabla S)^2|&\gg|\hbar\Delta S|
\end{align*}
Also folgt für kleine Wellenlängen
\[\frac{1}{2M}(\vec{\nabla}_{\vec x}S)^2+V(\vec x,t)=-\dif{S}{t}\]
Die allgemeine Lösung ist $\Psi(x,t=0)=Ae^{\frac{i}{\hbar}S_0(x)}:=\phi(x)$.\\
Mit der H-J-DGl bauen wir $A$ zu $A(\vec x,t)$ aus, sodass
\[\Psi(Q,t)=\sum_j\phi(q_0"j)\frac{1}{sqrt{\det(\dif{Q}{q_0})}_j}e^{\frac{i}{\hbar}S_j(Q,t)-\frac{i}{2}\pi\mu j}\]
Wobei $\mu$ den sog. Morse Index bezeichnet.

%Morseindex gibt Multiplizitaet in \dif{Q}{q_0} an (TODO genauer recherchieren)
%SOME NICE PLOTS HERE: P-Q-Phasendiagramm, x-V(x)-Potenzialbeispiel

\paragraph*{Grenzwert $\hbar$ gegen 0}
%TODO...? --> diesen Absatz nochmal ueberarbeiten, wegen schlechter Handschrift 
%Nachtrage Ende
Schroedingergleichung auf dimensionlosen Einteilen (???) bringen:
\begin{align*}
	 \left{-\frac{\hbar}{V_0^2 M q^2} \Delta \underbrace{\frac{\vec x}{q}}_{:=\vec S}
	 +\frac{V}{V_0}\left(\frac{\vec x}{q}\right)\right}\Psi(\vec x,t)\\
	=i\underbrace{\left[\frac{\hbar\omega_0}{V_0}\right]}_{:=1/\lambda} \dif{\Psi}{\underbrace{t \omega_0}_{:=\tau}}
\end{align*}
wobei $\lambda\gg 1$ dimensionslos ist. Weiter folgt:
\[\frac{1}{\lambda^2}=\frac{(\hbar\omega_0)}{V_0}^2\overset{!}{=}\frac{\hbar^2}{2mqV_0}\ll 1 \Leftrightarrow \frac{\hbar^2}{2mq}\ll V_0]
als Bedingung und damit ebenso $\omega^2=\sqrt{V_0/(2mq^2)}$. Damit ist die sogenante Lolemlisierungsenergie (??? furchtbare Handschrift ....) $\frac{\hbar^2}{2mq}$ vernachlässigbar.

\newpage

\section{Der starre Körper}
\subsection{Der starre Körper als Mechanisches Vielteilchensystem}
\paragraph*{Definition} Zwangsbedingungen: $|\vec{x}"i-\vec y"j|=C_{ij}$ $i,j\in\{1,...,N\}$

\begin{itemize}
	\item zeitunabhängig für alle Massepunkte
	\item Holonom
\end{itemize}

\begin{bem}
	\begin{itemize}
		\item Es gibt keine innere Dynamik
		\item kontinuierliche starre Körper durch kontinuums Limes\\
		Abstände bleiben im Körperfesten Bezugsystem Konstant.\\
		$3+3(N-3)=3N-6$ unabhängige Zwangsbedingungen.
		\item $N$ Massenpunkte $\rightarrow $ $3N$ Koordinaten für die Positionierung.\\
		Die Anzahl der Freien Parameter ist jedoch $3N-(3N-6)=6$ unabhängig von $N$
	\end{itemize}
\end{bem}


\subsection{Lagrangesche Bewegungsgleichungen 1. Art:}
\[S[\gamma]=\int_{t_0}^{t_1}dt\left\{\frac{m_i}{2}\dot{\vec{x}}_i^2(t)+\vec{F}_i(t)\cdot\vec x_i(t)+\sum_{l,m\in I}\la_{lm}(t)\big(|\vec{x}_l(t)-\vec{x}_x(t)|-c_{lm}\big)\right\}\]

\paragraph*{Bewegunsgleichungen}
\[\frac{d}{dt}m_i\dot{\vec{x}}_i(t)
	=\vec{F}_i(t)+\underbrace{\sum_{i,m\in I}\la_{im}(t)\frac{\vec x_I(t)
		-\vec x_m(t)}{|\vec x_i(t)-\vec x_m(t)|}
	-\sum_{l,i\in I}
	\frac{
		\vec x_l(t)-\vec x_i(t)}
	{|\vec x_l(t)-\vec x_i(t)|}}_{=\vec{Z}_i(t)}\]
\begin{itemize}
	\item %TODO
\end{itemize}

%VL 22.01.2016
\paragraph*{Kontaktbedingungen}
\begin{itemize}
	\item Gleiten: Zwangskraft steht orthogonal auf der Kontaktfläche
	\item Rollen: verschwindende Relativgeschwindigkeit 
\end{itemize}


\subsection{Lagrange Methode 2. Art}
\begin{figure}[h]
	\begin{center}
	\begin{tikzpicture}
		\coordinate (O) at (0,0,0);
		\coordinate (B) at (1,2,2);
		\draw[thick,->] (O)--(B)node[right]{$B$};
		\draw[fill=black](O) circle (0.05);
		\draw[fill=black](B) circle (0.05);
		\draw[->] (O)--(2,0,0) node[below] {$\vec n_2$};
		\draw[->] (O)--(0,2,0) node[above] {$\vec n_3$};
		\draw[->] (O)--(0,0,2) node[right] {$\vec n_1$};
		\draw[->] (B)--($(B)+(0,1,1)$) node [left]{$\vec e_1$};
		\draw[->] (B)--($(B)+(1,0,1)$) node[below] {$\vec e_2$};
		\draw[->] (B)--($(B)+(1,1,0)$) node[above] {$\vec e_3$};
		\draw (B) circle (1);
	\end{tikzpicture}
	\end{center}
	\caption{Freiheitsgrade im starren Körper}
	\label{fig:starr}
\end{figure}
\begin{itemize}
	\item 6 Freiheitsgrade
	\item Bezugssystem B im starren Körper $B\leftrightarrow\vec R(t)$\\
	entspricht der Position des Körpers
	\item $\{\vec e_1,\vec e_2,\vec e_3\}$ charakterisiert eien Körperfestes Orthonormalsystem \\
	$\Rightarrow\quad \vec x"i(t)=\vec R(t)+\sum_{r=1}^{3}b_r"i\vec e_r$\\
	\item Die Transformation $\{\vec e_1,\vec e_2,\vec e_3\}\leftrightarrow\{\vec n_1,\vec n_2,\vec n_3\}$ ist von 3 Parametern abhängig und eine Drehung.
	\item $\vec e_l=\sum_{r=1}^{3}\vec n_rD_{rl}(\vec \varphi(t))$. 
\end{itemize}


\paragraph*{Drehungen}Eine Drehung wir beschrieben, indem mit $|\vec{\varphi}|=\varphi$
\[D(\vec{\varphi}\vec x)=(\cos\varphi)\vec x+\sin\varphi\frac{\vec{\varphi}}{\varphi}\times\vec x+(1-\cos\varphi)\frac{\vec{\varphi}}{\varphi}(\frac{\vec{\varphi}}{\varphi}\cdot \vec x)\]
Sodass allgemein für $D(\varphi)$ gilt:
\[D(\vec{\varphi})=\cos\varphi\cdot\mathbb{1}+\sin\varphi\frac{\vec\varphi}{\varphi}+(1-\cos\varphi)\frac{\vec{\varphi}}{\varphi}\times\frac{\vec{\varphi}}{\varphi}\]
Charakteristische Eigenschaften: 

\begin{itemize}
	\item $D^{-1}(\vec \varphi)=D^T(\vec{\varphi})=D(-\varphi)$
	\item $\dot{\vec e}_l(t)=\sum_{r=1}^{3}\vec n_r\dot D_{rl}(\vec{\varphi})=\dot{\vec{\varphi}}\times\vec e_l:=\vec\omega$
\end{itemize}


\paragraph*{Parametrisierung durch Euler Winkel} $\vec{\al}\leftrightarrow(\varphi\theta,\psi)$\\
Wir zerlegen also in 3 Drehungen
\begin{itemize}
	\item $\{\vec n_i\}\overset{D"1}{\rightarrow }\{\vec{n_j}'\}$
	\item $\{\vec n_i'\}\overset{D"2}{\rightarrow }\{\vec{n_j}''\}$
	\item $\{\vec n_i''\}\overset{D"3}{\rightarrow }\{\vec e_l\}$
\end{itemize}
%TODO
\[D_{rl}(\varphi,\theta,\psi)=\begin{pmatrix}
\cos\varphi\cos\theta\cos\psi-\sin\varphi\sin\psi&
-\cos\varphi\sin\psi-\sin\varphi\cos\theta\cos\psi&
\sin\varphi\sin\theta\\
\sin\varphi\cos\psi+\cos\varphi\cos\theta\sin\psi&
-\sin\varphi\sin\psi+\cos\varphi\cos\theta\cos\psi&
-\cos\varphi\sin\theta\\
\sin\theta\sin\psi&
\sin\theta\cos\varphi&
\cos\theta
\end{pmatrix}\]


\subsubsection{Lagrange Funktion}
\[L(\vec R,\dot{\vec{R}},\vec{\Omega})=\sum_{i=1}^{N}\frac{m_i}{\dot{\vec x}}_i^2\]
Weiterhin gilt:
\begin{align*}
\vec x"i(t)=\vec R(t)+\sum_{r=1}^{3}b_r"i\vec e_r(t)
\dot{\vec x}"i(t)=\dot{\vec R}(t)+\sum_{r=1}^{3}b_r"i\underbrace{\dot{\vec{e}}_r(t)}_{=\vec \Omega(t)\times \vec e_r(t)}
\end{align*}
Sodass:
\begin{align*}
\sum_{i=1}^{N}\frac{m_i}{\dot{\vec x}}_i^2&=\frac{M}{2}\dot{\vec R}^2+(\dot{\vec{R}}\times\vec{\Omega})\cdot(\vec X_s-\vec R)M+\frac{1}{2}\sum_{k,l=1}^{3}\underbrace{(\vec{\Omega}\vec e_k)I_{kl}(\vec{\Omega}\vec e_l)}_{:=\Omega\cdot(I\vec\Omega)}
\end{align*}
Der sog Trägheisttensor $I$ wird definiert als
\[I:=\sum_{l,k=1}^{3}I_{kl}\vec{e}_k\times\vec e_l\]
Wobei $I_{kl}$ definiert ist als
\[I_{kl}=\sum_{i=1}^{N}m_i\left(\sum_{r=1}^{3}(b_r"i)^2\delta_{kl}-b_k"ib_l"i\right)\]
in körperfester Orthonormalbasis, zeitunabhängig. $I$ ist also nur abhängig von der Wahl von $\vec R$ und $\{\vec e_i\}$.\\
Die Trägheitsmatrix/ der Trägheitstensor kodiert die Struktur des Starren Körpers.
\paragraph*{Steinersche Satz}Für $I$ mit Bezugspunkt $\vec R$ und $I'$ mit Bp. $\vec R+\vec a$ gilt:
\[I_{kl}'=I_{kl}+M\left(\delta kl(\vec a)^2-(\vec a\vec e_k)(\vec a\vec e_l)\right)\]
%TODO VL 26.01.2016
%VL 28.01.2016
\subsection{Der freie starre Körper}
\begin{defn}
	Ein starrer Körper heißt \textbf{frei}, wenn nur die Zwangskräfte wirken.
\end{defn}
\begin{defn}
	Ein freier starrer Körper heißt \textbf{Kreisel}, wenn nur ein Punkt fest ist.
\end{defn}


\subsubsection{Erhaltungsgrößen}
\[L=\sum_{i=1}^{N}m"i\frac{{\dot{\vec{x}}"i}^2}{2}+\underbrace{\sum_{i,j}\la_{ij}\left(|\vec x"i-\vec x"j|-C_{ij}\right)}_{L_z}\]
\paragraph*{Translationen im Ort}
\begin{align*}
	\vec {x"i}'&=\vec x"i+\epsilon\vec a\\
	t'&=t\\
	L'&=L\\
	\intertext{Es folgt dass}
	\vec{P}_i&=\sum_{i=1}^{N}m_i\dot{\vec{x}}"i=M\dot{\vec X}_S\\
	E&=\sum_{j=1}^{3}\sum_{i=1}^{N}a_jm_i\dot{\vec x}_j"i=\vec a\vec P
	\intertext{Sodass}
	\dot{\vec P}&=0
\end{align*}

\paragraph*{Translation in der Zeit}
Sei $\phi=1$,$\psi_i=0$, $L=L'$. Dann
\[E=-\sum_{i=1}^{N}m_i\left(\frac{{\dot{\vec{x}}"i}^2}{2}-{\dot{\vec{x}}"i}^2\right)=\sum_{i=1}^{N}\frac{m_i}{2}{\dot{\vec{x}}"i}^2\]

\paragraph*{Veränderung der Geschwindigkeit}
Sei $\vec {x"i}'=\vec x"i+\epsilon\vec{v}t$. Wir bezeichnen $\vec v t$ als $\vec\psi$. Sei zusätzlich $t'=t$.
\[L'=\frac{1}{2}\sum_{i=1}^{N}m_i(\dot{\vec{x}}"i+\epsilon\vec v)^2+L_z=L+\epsilon\left(\sum_{i=1}^{N}m_i\dot{\vec x}"i\right)\]
%TODO


%VL 02.02.2016
\subsubsection{Dynamik des asymmetrischen starren Körpers}
wähle $\vec R=\vec X_s=\vec O$. und $I_{11}<I_{22}<I_{33}$
\begin{mth}
	Erhaltungsgrößen:
	\[\begin{array}{rl}
	E_{rot}&=\frac{1}{2}(\vec \Omega I\vec \Omega)=\sum_{i=1}^{3}\vec I_{ii}\vec{\Omega}_{ii}
	\end{array}\]
\end{mth}
%TODO


\subsubsection{Dynamik des freien Symmetrischen Kreisels}
%TODO
%VL 04.02.2016

\subsection[Der starre Körper im homogenen Schwerefeld]{Der starre Körper im homogenen Schwerefeld/unter dem Einfluss äußerer Kräfte}

\subsubsection{Rotation um eine raumfeste Drehachse mit konstanter Winkelgeschwindigkeit}
\paragraph{Problemstellung}Betrachte eine einfache Bewegung eines starren Körpers mit $\vec\Omega(t)=\vec{\Omega}_0\Rightarrow\dot{\Omega}(t)=\ddot{\Omega}(t)=...=0$.\\
Welche Kräfte sind zur Aufrechterhaltung der Bewegung erforderlich? (ohne Kräfte möglich?)\\
Sei $\{\vec e_j\}$ ein körperfestes Orthonormalsystem. Dann

\begin{align*}
	\vec x"i&=\vec R(t)+\sum_{j=1}^{3}b_j"i\vec e_j\\
	\dot{\vec x}"i(t)&=\dot{\vec R}(t)+\sum_{j=1}^{3}b_j"i
	\vec{\Omega}_0\times\vec e_j(t)\\
	\ddot{\vec x}"i(t)&=m_i\ddot{\vec R}(t)+\sum_{j=1}^{3}m_ij_j"i\vec{\Omega}_0\times(\vec{\Omega}_0\times\vec{e}_j(t))
\end{align*}
Es folgt für die Gesamtkraft
\begin{align*}
	\sum_{i=1}^{n}m_i\ddot{\vec x}_i&=M\ddot{\vec R}(t)+\vec{\Omega}_0\times\left(\sum_{j=1}^{N}\sum_{j=1}^{3}m_ib_j"i\vec e_j(t)\right)\\
	&=M\ddot{\vec R}(t)+\vec{\Omega}_0\times(\vec{\Omega}_o\times(\vec x_s-\vec R))
	\intertext{o.B.d.A: Sei $\vec R$ auf der Drehachse $\Rightarrow\dot{\vec{R}}=\ddot{\vec R}=0$}
	&=\vec{\Omega_0}\times(\vec{\Omega_0}\times(\vec x_s-\vec R))M\quad\text{Unwucht erster Art}
\end{align*}


\subparagraph*{Drehimpulsbilanz} $\vec R(t)=\vec x_s$
\begin{align*}
	\vec L&=M\vec x_s\times\vec x_s+I\vec{\Omega}\\
	\frac{d\vec L}{dt}&=M\vec x_s\times\ddot{\vec x}_s+\sum_{k=1}^{3}\vec{\Omega}_0\times\vec e_k(t)I_{kl}(\vec \Omega_0)_{l}=\vec N
\end{align*}
Sei die Unwucht 1. Art beseitigt ($\vec x_s$ auf Drehachse $\Leftrightarrow\ddot{\vec x}_s=0\Rightarrow\frac{d\vec P}{dt}=0$)\\
Dann ist
\begin{align*}
	\vec N&=\vec\Omega_0\times \vec L_{rel}&\vec L_{rel}&=I\vec{\Omega}_0\\
	\frac{d\vec L}{dt}&=0\Rightarrow \vec{\Omega}_0\parallel\vec L_{rel}&\vec L_{rel}&=I\vec{\Omega}_0=\alpha\vec\Omega_0
\end{align*}
Ist also eine Drehung um eine Hauptträgheitsachse


\subsubsection{Der schwere Kreisel}
Ein Kreisel im homogene Schwerefeld($V(\vec x"i)=-m_i\vec g\vec x"i$).\\
Lagrange Methode 2. Art:
\begin{align*}
	L&=\frac{M}{2}\dot{\vec{x}}_s^2+\dfrac{1}{2}(\vec{\Omega},I\vec{\Omega})-\sum_{i=1}^{N}V(\vec x"i)+\text{Zwbedg}
	\intertext{Entlang der Hauptträgheistachsen}
	&(\vec{\Omega},I\vec{\Omega})=\sum_{j=1}^{3}\Omega_j I_{jj}\Omega_j
\end{align*}
verwende Eulerwinkel $(\varphi,\theta,\psi)$
\[L=\frac{1}{2}(I_{11}]M{b_3"P}^2)(\dot\theta^2+\dot\varphi^2\sin^2\theta^2)+I_{33}(\dot{\psi}^2+\dot{\varphi}^2\cos^2\theta+2\dot\varphi\dot{\psi}\cos\theta)+Mgb_3"P\cos\theta-Mg\cos\theta-Mg\vec n_3\vec C
\]
Es folgt:
\begin{align*}
	\dif{L}{\varphi}&=0&\dif{L}{\dot{\varphi}}&=(I_{11}+M{b_3"P}^2)\sin^2\theta\dot \varphi+I_{33}\cos^2\theta\dot \varphi+I_{33}\dot \psi \cos\theta&&:=A=\vec L\vec n_3\\
	\dif{L}{\psi}&=0&\dif{L}{\dot{\psi}}&=I_{33}(\dot \psi +\dot{\varphi}\cos\theta)&&:=B=\vec L\vec e_3\\
	\dif{L}{t}&=0&\sum_{j=1}^{3}&\dif{L}{\dot q_j}\dot q_j-L&&:=E=L_2-L_1
\end{align*}
Einsetzen der Erhaltungsgrößen in $L$:
\begin{align*}
	\dot{\psi}&=\frac{B}{I_{33}}-\dot{\varphi}\cos\theta&
	\dot\varphi&=\frac{A-B\cos\theta}{\sin^2\theta[I_{11}+{b_3"P}^2]}
\end{align*}
\begin{align*}
	E&=\frac{1}{2}(I_{11}+M{b_3"P}^2\dot{\theta}^2+U_{eff}(\theta))\\
	U_{eff}(\theta)&=\frac{(A-B\cos\theta)^2}{2(I_{11}+M{b_3"P}^2)\sin^2\theta}+\frac{B^2}{2I_{33}}-Mgb_3"P\cos\theta
\end{align*}


%VL 09.02.2016
\paragraph*{...}
\begin{itemize}
	\item $\theta(t)=\theta_0$, $\dot{\varphi}=\dfrac{A-B\cos\theta_0}{\sin^2\theta_0I'_{11}}$:\\
	reine Präzessionsbewegung, keine Nutation
	\item $U_{eff}(\theta)=U_{eff}(\theta_0)+\dfrac{1}{2}(\theta-\theta_0)^2U_{eff}''(\theta_0)+O\big((\theta-\theta_0)^3\big)$,\\
	$E=\frac{1}{2}I'_{11}\dot\theta^2+\frac{1}{2}(\theta-\theta_0)^2U_{eff}''(\theta_0)\rightarrow $ harmonische Schwingung um $\theta_0$
\end{itemize}

\paragraph*{Spezialfall $\theta\approx0$} Gibt es eine Beewgung mit $\theta(t)=0$?\\
Existiert nur für spezielle Werte von $A,B$.\\
Sei $A=B,|\theta|\ll 1$

\[U_{eff}(\theta)\rightarrow \frac{(A-B(1-\frac{\theta}{2}))}{2I'_{11}\theta^2}+\frac{B^2}{2I_{33}}-Mgb_3"P(1-\frac{\theta^2}{2})+O(\theta^3)
=\left(\frac{B^2}{2I_{11}'}\frac{1}{4}+Mgb_3"P\frac{1}{2}\right)\theta^2+\frac{B^2}{2I_{33}}-Mgb_3"P\]

Es ergibt sich eien Harmonsiche Schwingung mit $\omega>0$:\\

$\omega^2:=\frac{\frac{B^2}{8I_{11}'}+\frac{Mg}{2}b_3"P}{\frac{1}{2}I_{11}'}$, $B=\vec L\vec n_3=\vec L\cdot \vec e_3$\\

Eine instaile Dynamik für $\omega^2<0$ mit $b_3"P=-|b_3"P|$:\\

$\frac{B^2}{8I_{11}'}<\frac{Mg}{2}|b_3"P|$
\end{document}
